\documentclass[a4paper,11pt]{article}
% Various packages
\usepackage{siunitx}
\usepackage[utf8]{inputenc} % æøå
\usepackage[T1]{fontenc} % mere æøå
\usepackage[danish]{babel} % orddeling
\usepackage{verbatim} % så man kan skrive ren tekst
\usepackage{graphicx}
\graphicspath{{assets/}}
\usepackage{a4wide}
\usepackage{url}
\usepackage[left=2cm,top=2cm,bottom=1.5cm,right=2cm]{geometry}
\usepackage{amsmath}
\usepackage{amssymb}
\usepackage{amsthm}
\usepackage{wrapfig}
\usepackage{fixme}
\usepackage{color}
\usepackage[makeroom]{cancel}
\usepackage{pstricks}
\usepackage{pdfpages} % include pdf
\usepackage{forest}
\usepackage{float} % Use [H] in figures
\usepackage{subcaption} % For subfigures
\usepackage{color} % May be necessary if you want to color links
%\usepackage{xcolor} \pagecolor[rgb]{0,0,0} \color[rgb]{1,1,1}
\usepackage{nameref}
\usepackage{hyperref} % Make references clickable
\usepackage[nameinlink,capitalize]{cleveref} % Make eq:refs be in style (1)
\usepackage[linesnumbered, commentsnumbered, lined, ruled, vlined,
%noend  % Have no ⌊-like symbol to indicate end of scope in pseudocode
]{algorithm2e} % Doc: https://goo.gl/6bC1qZ

\crefname{equation}{}{Equations}

% Ændr på navnene der vises når man bruger \autoref{label}
\def\sectionautorefname{Sektion}
\renewcommand{\equationautorefname}{Ligning}
\def\figureautorefname{Figur}
\AtBeginDocument{\renewcommand{\ref}[1]{\autoref{#1}}}

% Sæt \ref{} til at kalde \autoref{}
\AtBeginDocument{\renewcommand{\ref}[1]{\autoref{#1}}}

% Ændr ''*'' i math-felter til \cdot
\DeclareMathSymbol{*}{\mathbin}{symbols}{"01}

% Sæt farver for interne referencer og links
\definecolor{darkblue}{RGB}{25,25,112}
\hypersetup{
	colorlinks=true,    %set true if you want colored links
	linktoc=all,        %set to all if you want both sections and subsections linked
	linkcolor=darkblue, %choose some color if you want links to stand out
	filecolor=blue,     %
	citecolor=black,    %
	urlcolor=cyan,      %
}

% Set indentation to 0:
\setlength\parindent{0pt}

% Keywords relateret til algorithm2e pakken
\newcommand{\True}{\textbf{true}}\newcommand{\False}{\textbf{false}}
\SetStartEndCondition{ }{}{}%
\SetKwProg{Fn}{def}{\string:}{}
\SetKw{KwTo}{to}
\SetKwFor{For}{for}{}{}%
\SetKwFor{ForEach}{foreach}{}{}%
\SetKwIF{If}{ElseIf}{Else}{if}{}{elif}{else}{end}%
\SetKwFor{While}{while}{}{end}\SetKwProg{Fn}{}{}{}
\SetKwInOut{Input}{input}\SetKwInOut{Output}{output}
\setlength{\algomargin}{3em}\DontPrintSemicolon

\newcommand{\longspace}{{\ \ \ \ \ \ \ \ \ \ \ \ \ \ }}
% \renewcommand{\P}{{\mathbb P}}
\newcommand{\R}{{\mathbb R}}
\newcommand{\E}{{\mathbb E}}
\newcommand{\event}{{\mathcal{E}}}
\newcommand{\parfrac}[1]{\frac{\partial}{\partial #1}}
\renewcommand{\num}{{\textrm{num} }}
\newcommand{\size}{{\textrm{size} }}
\newcommand{\ift}{{\textrm{if } }}

\newcommand{\pfrac}[2]{\left( \frac{#1}{#2} \right)}

% Dynamiske (), <>, ceil, floor
\newcommand{\p}[1]{\left( #1 \right)}
\newcommand{\pbig}[1]{\big( #1 \big)}
\newcommand{\pBig}[1]{\Big( #1 \Big)}
\newcommand{\pbigg}[1]{\bigg( #1 \bigg)}
\newcommand{\curly}[1]{\left\{ #1 \right\}}
\renewcommand{\square}[1]{\left[ #1 \right]}
\newcommand{\larr}[1]{\left< #1 \right>}
\newcommand{\ceil}[1]{\left\lceil #1 \right\rceil}
\newcommand{\floor}[1]{\left\lfloor #1 \right\rfloor}

\renewcommand{\P}[1]{\mathbb{P} \square{ #1 } }

\author{Søren Mulvad, rbn601}

\title{Eksamensdisposition - Heavy Hitters}

\begin{document}
\maketitle



%%%%%%%%%%%%%%%%%%%%%%%%%%%%%%%%%%%%%%%%%%%%%%%%%%%%%%%%%%%
%%%%%%%%%%%%%%%%%%%%%%%%%%%%%%%%%%%%%%%%%%%%%%%%%%%%%%%%%%%
%%%%%%%%%%%%%%%%%%%%%%%%%%%%%%%%%%%%%%%%%%%%%%%%%%%%%%%%%%%
\newpage
%%%%%%%%%%%%%%%%%%%%%%%%%%%%%%%%%%%%%%%%%%%%%%%%%%%%%%%%%%%
%%%%%%%%%%%%%%%%%%%%%%%%%%%%%%%%%%%%%%%%%%%%%%%%%%%%%%%%%%%
%%%%%%%%%%%%%%%%%%%%%%%%%%%%%%%%%%%%%%%%%%%%%%%%%%%%%%%%%%%
\section{Eksamensdisposition - Heavy Hitters}
% Sektion 4

\subsection{Problem}

Vi har en strøm af par $(j_0, \Delta_0), \dots, (j_{s-1}, \Delta_{s-1}) \in [n]\times \mathbb Z$ (sættet af alle heltal).

For hver eneste $j \in [n]$, lad $I_j = \{ i \in [n] \ | \ j_i = j \}$ og definer frekvensen
$$
f_j = \sum_{i \in I_j} \Delta_i
$$

Nu ønsker vi givet et $a$ at beregne et estimat $\hat f_a$ for $f_a$.

\subsection{Algoritme}

\begin{algorithm}[H] \caption{Basic Count Sketch} \label{alg:bcs}
  \SetKwFunction{zeros}{zeros}%
  %\BlankLine
  \nonl Init\;
  $k = \ceil{\frac{4}{\epsilon^2}}$\;
  $C[0, \dots, k-1] = 0$\;
  $h =$ strong-universel $h : [n] \rightarrow [k]$\;
  $s =$ strong-universel $s : [n] \rightarrow \{-1, +1 \}$\;
  \nonl Process $(j, \Delta)$\;
  $C[h(j)]$ += $s(j) * \Delta$\;
  \nonl Output\;
  \Return $\hat f_a = s(a) * C[h(a)]$
\end{algorithm}\vspace{1em}

Hvor Process $(j, \Delta)$ svarer til vi løbende kører alle par i strømmen igennem i linje 5, hvor vi potentielt kunne stoppe på et vilkårligt tidspunkt.\\

\subsection{Analyse}
Lad os fiksere en nøgle $a$ og betragte outputtet $X = \hat f_a$ for en query $a$.

For enhver nøgle $j \in [n]$, definer da indikatorvariablen $Y_j = [h(j) = h(a)]$.

Vi ser, at en nøgle $j$ bidrager til tælleren $C[h(a)]$ hvis og kun hvis $h(j) = h(a)$.

Mængden den bidrager med er den frekvens $f_j$ ganget med den tilfældige fortegn $s(j)$. Derfor:

\begin{align}
  X = \hat f_a
  &= s(a) * C[h(a)] \nonumber \\
  &= s(a) \sum_{j \in [n]} f_j \, s(j) Y_j \label{eq:sa-cha} \\
  &= f_a \, s(a) s(a) Y_a + s(a) \summ{j \in [n]\\ j \neq a} f_j \, s(j) Y_j \label{eq:split-i-to} \\
  &= f_a + s(a) \summ{j \in [n]\\ j \neq a} f_j \, s(j) Y_j \label{eq:temp-res}
\end{align}

I \cref{eq:sa-cha} benytter vi at et element kun tælles med når $Y_j = 1$.\\
I \cref{eq:split-i-to} splitter vi vores sum i to, så vi tager højde for den unikke case når $j=a$ og alle andre cases.\\
I \cref{eq:temp-res} benytter vi $s(a)s(a) = 1*1$ eller $(-1)(-1) = 1$ og $Y_a = [h(a) = h(a)] = 1$.\\


Vi kan da regne på den forventede værdi af udtrykket i sumtegnet i \cref{eq:temp-res}:
\begin{align}
  \E{f_j \, s(j) Y_j}
  = f_j \ \underbrace{\E{s(j)}}_{0} \ \E{Y_j}
  = 0 \label{eq:exp-x}
\end{align}

Hvor forventningen af produktet er lig produktet af de forskellige forventninger da $s$ er 2-uafhængig, og $s$ og $h$ er uafhængige af hinanden.\\

Da kan vi bruge \cref{eq:exp-x} til at regne videre på \cref{eq:temp-res} hvorved vi får:
\begin{align}
  \E{X}
  = f_a + s(a) \summ{j \in [n]\\ j \neq a} \E{f_j \, s(j) Y_j}
  = f_a
\end{align}

Hermed har vi altså vist at $X = \hat f_a$ er en unbiased estimator for frekvensen $f_a$. Men vi skal stadig vise at det er usandsynligt den afviger for meget fra dens forventede værdi.\\


Derfor analyserer vi dens varians:
\begin{align*}
  \Var{X}
  &= \E{(\hat f_a - f_a)^2}\\
  &= \E{\pbigg{ f_a + s(a) \summ{j \in [n]\\ j \neq a} f_j \, s(j) Y_j - f_a}^2 }\\
  &= \E{\pbigg{ s(a) \summ{j \in [n]\\ j \neq a} f_j \, s(j) Y_j }^2 }\\
  &= \underbrace{(s(a))^2}_{1} \ \summ{i, j \in [n]\\i \neq a, \; j \neq a} \E{ f_i f_j s(i) s(j) Y_i Y_j }
\end{align*}

Vi bruger nu, at $h$ er stærk universel, så for ethvert $j \in [n]$ hvor $j \neq a$ har vi:
$$
\E{Y_j^2} = \E{Y_j} = \P{h(j) = h(a)} = \frac{1}{k}
$$
da $Y_j = 0 \lor 1$ og vi har $0^2 = 0$ og $1^2 = 1$.\\

Hvis vi kigger på udtrykket i summen, så har vi:
$$
\E{ f_i f_j s(i) s(j) Y_i Y_j } =
\begin{cases}
	f_j^2 \underbrace{(s(j))^2}_{1} \underbrace{\E{Y_j^2}}_{\E{Y_j}} = f_j^2 / k & i = j\\
  f_i f_j \underbrace{\E{s(i)}}_{0} \underbrace{\E{s(j)}}_{0} \E{Y_i Y_j} = 0 & i \neq j
\end{cases}
$$

Vi definerer $\sum_{j \in [n]} f_j^2 = || \mathbf f ||_2^2$ (udtales ''to-normen i anden''), hvorved vi kan regne ud vores udtryk bliver (hvor $|| \mathbf f_{-a} ||_2^2 = || \mathbf f ||_2^2 - f_a^2$.):
$$
\Var{X} = \summ{j \in [n]\\ j \neq a} \frac{f_j^2}{k} = \frac{|| \mathbf f_{-a} ||_2^2}{k}
$$

Med vores informationer for forventning og varians kan vi nu benytte Chebyshev:
$$
\P{|\hat f_a - f_a| \geq \epsilon || \mathbf f_{-a} ||_2}
\leq \frac{\Var{X}}{\epsilon^2 ||\mathbf f_{-a}||_2^2}
= \frac{1}{k \epsilon^2}
\leq \frac{1}{4}
$$
Idet vi satte $k = \ceil{4 / \epsilon^2}$ i vores algoritme.


\subsection{Median trick}
Vi laver $t$ uafhængige estimater $X_0, \dots, X_{t-1}$ i parallel (ved at bruge forskellige hashfunktioner) og returnerer medianen $X_{(\ceil{t/2})}$ af de $t$ svar (Tal om hvad man skulle ændre i algoritmen).\\

Vi siger $X_i$ fejler hvis $|X_i - \E{X}| \geq Q$, hvor $Q$ f.eks. er vores sandsynlighed fra før.\\
Lad $B_i = [X_i \text{ fejler}]$ og lad $B$ være antallet der fejler:
$$
B = \sum_{i \in [t]} B_i
$$


Da har vi, at hvis $X_{(\ceil{t/2})}$ fejler, så betyder det at $B \geq t/2$.\\
Den forventede værdi af $B$ må være
$$
\E{B} = \mu = \sum_{i \in [t]} \E{B_i} \leq t/4
$$

Da kan vi beregne:
$$
\P{\text{Median fejler}}
= \P{B \geq 2 \mu}
= \P{B \geq (1 + \delta) \mu }
\leq e^{- \delta^2 \mu / 3}
\leq e^{-t/12}
$$

Hvor vi bruger Chernoff Bounds til at begrænse sandsynligheden, og herudover i eksponentialet benytter $\delta = 1$.



Herved har vi altså væsentligt begrænset sandsynligheden for at få noget rimelig forkert.

\end{document}
