\documentclass[a4paper,11pt]{article}
% Various packages
\usepackage{siunitx}
\usepackage[utf8]{inputenc} % æøå
\usepackage[T1]{fontenc} % mere æøå
\usepackage[danish]{babel} % orddeling
\usepackage{verbatim} % så man kan skrive ren tekst
\usepackage{graphicx}
\graphicspath{{assets/}}
\usepackage{a4wide}
\usepackage{url}
\usepackage[left=2cm,top=2cm,bottom=1.5cm,right=2cm]{geometry}
\usepackage{amsmath}
\usepackage{amssymb}
\usepackage{amsthm}
\usepackage{wrapfig}
\usepackage{fixme}
\usepackage{color}
\usepackage[makeroom]{cancel}
\usepackage{pstricks}
\usepackage{pdfpages} % include pdf
\usepackage{forest}
\usepackage{float} % Use [H] in figures
\usepackage{subcaption} % For subfigures
\usepackage{color} % May be necessary if you want to color links
%\usepackage{xcolor} \pagecolor[rgb]{0,0,0} \color[rgb]{1,1,1}
\usepackage{nameref}
\usepackage{hyperref} % Make references clickable
\usepackage[nameinlink,capitalize]{cleveref} % Make eq:refs be in style (1)
\usepackage[linesnumbered, commentsnumbered, lined, ruled, vlined,
%noend  % Have no ⌊-like symbol to indicate end of scope in pseudocode
]{algorithm2e} % Doc: https://goo.gl/6bC1qZ

\crefname{equation}{}{Equations}

% Ændr på navnene der vises når man bruger \autoref{label}
\def\sectionautorefname{Sektion}
\renewcommand{\equationautorefname}{Ligning}
\def\figureautorefname{Figur}
\AtBeginDocument{\renewcommand{\ref}[1]{\autoref{#1}}}

% Sæt \ref{} til at kalde \autoref{}
\AtBeginDocument{\renewcommand{\ref}[1]{\autoref{#1}}}

% Ændr ''*'' i math-felter til \cdot
\DeclareMathSymbol{*}{\mathbin}{symbols}{"01}

% Sæt farver for interne referencer og links
\definecolor{darkblue}{RGB}{25,25,112}
\hypersetup{
	colorlinks=true,    %set true if you want colored links
	linktoc=all,        %set to all if you want both sections and subsections linked
	linkcolor=darkblue, %choose some color if you want links to stand out
	filecolor=blue,     %
	citecolor=black,    %
	urlcolor=cyan,      %
}

% Set indentation to 0:
\setlength\parindent{0pt}

% Keywords relateret til algorithm2e pakken
\newcommand{\True}{\textbf{true}}\newcommand{\False}{\textbf{false}}
\SetStartEndCondition{ }{}{}%
\SetKwProg{Fn}{def}{\string:}{}
\SetKw{KwTo}{to}
\SetKwFor{For}{for}{}{}%
\SetKwFor{ForEach}{foreach}{}{}%
\SetKwIF{If}{ElseIf}{Else}{if}{}{elif}{else}{end}%
\SetKwFor{While}{while}{}{end}\SetKwProg{Fn}{}{}{}
\SetKwInOut{Input}{input}\SetKwInOut{Output}{output}
\setlength{\algomargin}{3em}\DontPrintSemicolon

\newcommand{\longspace}{{\ \ \ \ \ \ \ \ \ \ \ \ \ \ }}
% \renewcommand{\P}{{\mathbb P}}
\newcommand{\R}{{\mathbb R}}
\newcommand{\E}{{\mathbb E}}
\newcommand{\event}{{\mathcal{E}}}
\newcommand{\parfrac}[1]{\frac{\partial}{\partial #1}}
\renewcommand{\num}{{\textrm{num} }}
\newcommand{\size}{{\textrm{size} }}
\newcommand{\ift}{{\textrm{if } }}

\newcommand{\pfrac}[2]{\left( \frac{#1}{#2} \right)}

% Dynamiske (), <>, ceil, floor
\newcommand{\p}[1]{\left( #1 \right)}
\newcommand{\pbig}[1]{\big( #1 \big)}
\newcommand{\pBig}[1]{\Big( #1 \Big)}
\newcommand{\pbigg}[1]{\bigg( #1 \bigg)}
\newcommand{\curly}[1]{\left\{ #1 \right\}}
\renewcommand{\square}[1]{\left[ #1 \right]}
\newcommand{\larr}[1]{\left< #1 \right>}
\newcommand{\ceil}[1]{\left\lceil #1 \right\rceil}
\newcommand{\floor}[1]{\left\lfloor #1 \right\rfloor}

\renewcommand{\P}[1]{\mathbb{P} \square{ #1 } }

\author{Søren Mulvad, rbn601}

\title{Eksamensdisposition - Kapitel 4}

\begin{document}
\maketitle

\begin{itemize}
\item \textbf{Chernoff bounds}
\begin{itemize}
  \item Bevis for case med $\P{X > (1 + \delta) \mu}$
  \item Bevis for case med $\P{X < (1 - \delta) \mu}$
\end{itemize}

\item \textbf{Set balancing}
\begin{itemize}
  \item Håndkørsel af algoritme
  \item Bevis for sandsynlighed for tilpas afvigende resultat
\end{itemize}
\end{itemize}


%%%%%%%%%%%%%%%%%%%%%%%%%%%%%%%%%%%%%%%%%%%%%%%%%%%%%%%%%%%
%%%%%%%%%%%%%%%%%%%%%%%%%%%%%%%%%%%%%%%%%%%%%%%%%%%%%%%%%%%
%%%%%%%%%%%%%%%%%%%%%%%%%%%%%%%%%%%%%%%%%%%%%%%%%%%%%%%%%%%
\newpage
%%%%%%%%%%%%%%%%%%%%%%%%%%%%%%%%%%%%%%%%%%%%%%%%%%%%%%%%%%%
%%%%%%%%%%%%%%%%%%%%%%%%%%%%%%%%%%%%%%%%%%%%%%%%%%%%%%%%%%%
%%%%%%%%%%%%%%%%%%%%%%%%%%%%%%%%%%%%%%%%%%%%%%%%%%%%%%%%%%%
\section{Eksamensdisposition - Kapitel 4}

\section{Chernoff Bounds}
\subsection{Case for $\P{X > (1 + \delta) \mu}$}
Givet uafhængige Poisson trials $X_1, \dots, X_n$, dvs. for $i = 1, \dots, n$, lad $X_i$ være en indikatorvariabel med $\P{X_i = 1} = p_i$ hvor $0 < p_i < 1$. Lad
$$
\delta > 0
\quad\quad\quad\quad
X = \sum_{i=1}^n X_i
\quad\quad\quad\quad
\mu = \mu_X = \sum_{i=1}^n p_i
$$
Da gælder:
\begin{align*}
  \P{X > (1+\delta) \mu} < \pfrac{e^\delta}{(1+\delta)^{1 + \delta}}^\mu
\end{align*}


\subsubsection{Bevis}
Lad $t > 0$ (fastlægges senere til $t = \ln(1 + \delta)$). Da gælder:
\begin{align}
  \P{X > (1+\delta) \mu}
  &= \P{e^{tX} > e^{t(1+\delta) \mu}} \label{eq:i-ete} \\
  &< \frac{\E{e^{tX}}}{e^{t(1+\delta) \mu}} \label{eq:bruger-markov}
\end{align}
I \cref{eq:i-ete} opløfter vi $e$ med vores udtryk, hvorved uligheden stadig gælder og i \cref{eq:bruger-markov} bruger vi Markovs ulighed. Lad os nu regne videre på nævneren i \cref{eq:bruger-markov}:
\begin{align}
  \E{e^{tX}}
  &= \E{\prod_{i=1}^n e^{t X_i}} \label{eq:stok-var}\\
  &= \prod_{i=1}^n \E{e^{t X_i}} \label{eq:uafhaang}\\
  &= \prod_{i=1}^n \p{p_i e^t + (1-p_i) e^0 } \label{eq:brug-exp}\\
  &= \prod_{i=1}^n \p{1 + p_i (e^t - 1)} \label{eq:reducer}\\
  &\leq \prod_{i=1}^n e^{p_i (e^t - 1)} \label{eq:brug-1+x-regel}\\
  &= e^{\p{\sum_{i=1}^n p_i (e^t - 1)}} \label{eq:prod-til-sum} \\
  &= e^{\mu (e^t - 1)} \label{eq:beregn-sum}
\end{align}

I \cref{eq:stok-var} bruger vi definitionen for $X$, og bestemmer produktet i stedet for summen da den står i en potens.
I \cref{eq:uafhaang} benytter vi at vores stokastiske variable er uafhængige, og derfor kan vi flytte vores forventning ind i produktet.\\
I \cref{eq:brug-exp} benytter vi definitonen på forventningsværdien og egenskaberne for Poisson trials.\\
I \cref{eq:reducer} reducerer vi blot udtrykket.\\
I \cref{eq:brug-1+x-regel} benytter vi reglen $1 + x \leq e^x \ \forall x \in \R$.\\
I \cref{eq:prod-til-sum} har vi, at produktet af en masse udtryk med samme base svarer til summen af alle potenserne.\\
I \cref{eq:beregn-sum} benytter vi vores definition af $\mu$.\\

Indsætter vi denne værdi tilbage i \cref{eq:bruger-markov} får vi:

\begin{align}
  \frac{\E{e^{tX}}}{e^{t(1+\delta) \mu}}
  &\leq \frac{e^{\mu (e^t - 1)}}{e^{t(1+\delta) \mu}}\\
  &= \frac{e^{\delta \mu}}{(1 + \delta)^{(1 + \delta) \mu}} \label{eq:indsaat-t} \\
  &= \pfrac{e^\delta}{(1 + \delta)^{1 + \delta}}^\mu \label{eq:reducer-potens}
\end{align}

I \cref{eq:indsaat-t} indsætter vi vores $t = \ln(1 + \delta)$ så vores udtryk bliver pænt.\\
I \cref{eq:reducer-potens} sætter vi blot $\mu$ uden for en parentes for brøken.



\subsection{Case for $\P{X < (1 - \delta) \mu}$}
Givet uafhængige Poisson trials $X_1, \dots, X_n$, dvs. for $i = 1, \dots, n$, lad $X_i$ være en indikatorvariabel med $\P{X_i = 1} = p_i$ hvor $0 < p_i < 1$. Lad
$$
0 < \delta \leq 1
\quad\quad\quad\quad
X = \sum_{i=1}^n X_i
\quad\quad\quad\quad
\mu = \mu_X = \sum_{i=1}^n p_i
$$
Da gælder:
\begin{align}
  \P{X < (1 - \delta) \mu} < e^{- \frac{\delta^2 \mu}{2}} \label{eq:chernoff-2}
\end{align}


\subsubsection{Bevis}
Lad $t > 0$ (fastlægges senere til $t = - \ln(1 - \delta)$). Da gælder:
\begin{align}
  \P{X < (1 - \delta) \mu}
  &= \P{e^{- t X} > e^{-t (1 - \delta) \mu} } \label{eq:ref1} \\
  &= \pfrac{e^{-\delta}}{(1 - \delta)^{1 - \delta}}^\mu \nonumber
\end{align}

Hvor vi gør stort set det samme som før, men havde den lidt anderledes start i \cref{eq:ref1} og derudover brugte $t = -\ln(1 - \delta)$. Lad os nu regne videre på dette:


\begin{align}
  \pfrac{e^{-\delta}}{(1 - \delta)^{1 - \delta}}^\mu
  &= \pfrac{e^{-\delta}}{e^{(1 - \delta) \ln (1 - \delta)}}^\mu \label{eq:omskriver} \\
  &< \pfrac{e^{- \delta}}{e^{- \delta + \frac{1}{2} \delta^2}}^\mu \label{eq:mclaurin} \\
  &= e^{- \frac{1}{2} \delta^2 \mu} \label{eq:pot-regne-regler}
\end{align}

I \cref{eq:omskriver} omskriver vi blot for at få noget med $e$ i nævneren. I \cref{eq:mclaurin} benytter vi McLaurin expansion, som medfører at $(1 - \delta) \ln(1 - \delta) > - \delta + \frac{1}{2} \delta^2$. Og endelig i \cref{eq:pot-regne-regler} er det blot simple potensregneregler.

\subsubsection{McLaurin Expansion}
Givet en funktion $f(x) : \R \to \R$, som er vilkårligt ofte differentiabel i $x = 0$, da gælder:
$$
f(x) = \sum_{i=0}^n \frac{f^{(i)} (0)}{i!} x^i
$$
hvor f.eks. $f^{(3)} (0)$ betegner $f'''(0)$.\\

For vores konkrete case ovenfor betyder det:
$$
(1 - \delta) \ln(1 - \delta)
= - \delta + \frac{1}{2} \delta^2 + \frac{1}{6} \delta^3 + \frac{1}{12} \delta^4 + \frac{1}{20} \delta^5 + \dots > - \delta + \frac{1}{2} \delta^2
$$






\section{Set Balancing}
\subsection{Problem}

Givet en $n \times n$ $0/1$-matrix $\mathbf A$, find vektor $\mathbf b \in \curly{-1, 1}^n$ så $||\mathbf{Ab}||_\infty$ (udtales infinity-normen) bliver så lille som mulig. Eksempelvis kunne vi have:
\begin{alignat*}{5}
  &\begin{bmatrix}
  0 & 1 & 1\\
  1 & 0 & 1\\
  1 & 1 & 1
  \end{bmatrix}
  &&\begin{bmatrix}
  -1\\
  1\\
  1
  \end{bmatrix}
  =
  &&\begin{bmatrix}
  2\\
  0\\
  1
  \end{bmatrix}
  \quad\quad &&\rightarrow \quad\quad
  &&||\mathbf{Ab}||_\infty = 2\\
  %%%%%%%%%%%%%%%%%%%%%%%
  &\quad\quad \mathbf A
  &&\quad \ \mathbf b
  && \ \mathbf{Ab}\\
  %%%%%%%%%%%%%%%%%%%%%%%
  &
  &&\begin{bmatrix}
  -1\\
  -1\\
  1
  \end{bmatrix}
  =
  &&\begin{bmatrix}
  0\\
  0\\
  -1
  \end{bmatrix}
  \quad\quad &&\rightarrow \quad\quad
  &&||\mathbf{Ab'}||_\infty = 1\\
  %%%%%%%%%%%%%%%%%%%%%%%
  &
  &&\quad \ \mathbf b'
  && \ \mathbf{Ab'}\\
\end{alignat*}

Her er f.eks. $||\mathbf{Ab}||_\infty = 2$ da den er defineret som den numerisk største værdi i vektor $\mathbf{Ab}$.\\

Man kan f.eks. forstå problemet som at matrix $\mathbf A$ i søjlerne har en række individer, og i rækkerne har en værdi for om besidder en bestemt egenskab eller ej. Så vil vi gerne dele individerne i to grupper, som skal sikre så jævn fordeling af egenskaberne som muligt. $||\mathbf{Ab}||_\infty$ er da et udtryk for den egenskab der er mest ujævnt fordelt.






\subsection{Algoritme}
Vælg indgang $i$ i $\mathbf b$ til $-1$ med sandsynlighed $1/2$ og ellers $1$ for alle $i$ uafhængigt.






\subsection{Analyse af sandsynlighed for $||\mathbf{Ab}||_\infty \leq 2 \sqrt{2 n \ln n}$}
Betragt række $j$ i $\mathbf A$, $\mathbf A_j$. Antag at $\mathbf A_j$ er på følgende form, som er en gyldig måde at anskue det på så længe vi kun kigger på én række, da vi i vores analyse bare kunne flytte alle 1-taller i $\mathbf A_j$ til venstre og så have de resterende 0-taller til højre:
$$
\mathbf A_j = [ \; \underbrace{1 \; \dots \; 1}_{k} \; \underbrace{0 \; \dots \; 0}_{n-k} \; ]
$$

Alle 0-taller kan vi da ignorere, da de ikke vil bidrage til summen i vores analyse.\\
For $i = 1, \dots, k$, lad da indikatorvariablen $X_i$ være
$$
X_i
=
\begin{cases}
	1 & \text{hvis $b_i = -1$}\\
	0 & \text{ellers (dvs. $b_i = 1$)}\\
\end{cases}
$$

Lad nu
$$
X = \sum_{i=1}^k X_i \quad\quad\quad\quad \mu_X = k * \P{X_i = 1} = \frac{k}{2}
$$


Vi får $\mathbf A_j \mathbf b = 0$ når de første $k$ indgange i $\mathbf b$ er præcis ligeligt fordelte mellem at være $-1$ og $1$:

\[
  \setlength{\arraycolsep}{0pt}
  \mathbf{b} =
    \left[
      \begin{array}{c}
        \begin{array}{c}
          -1 \\ -1 \\ 1 \\ 1
        \end{array} \\
        \begin{array}{c}
          \vdots
        \end{array}
      \end{array}
    \right]
  ~ % Some space
  \begin{array}{c}
    \noleftdelimiter
    \vphantom{\begin{array}{c}
      -1 \\ -1 \\ 1 \\ 1
    \end{array}}
    \right\} k \\
    %\noleftdelimiter
    \vphantom{\begin{array}{c}
      \vdots
    \end{array}}
    %\right\} n - k
  \end{array}
\]


Altså får vi:

\begin{align}
  \P{\mathbf A_j \mathbf b = 0} &= \P{X = \frac{k}{2}} \nonumber \\
  \P{\mathbf A_j \mathbf b = 2} &= \P{X = \frac{k}{2} - 1} \nonumber \\
  &\vdots \nonumber \\
  \P{\mathbf A_j \mathbf b = 2c} &= \P{X = \frac{k}{2} - c} \nonumber \\
  &\Updownarrow \nonumber \\
  \P{\mathbf A_j \mathbf b > 2c} &= \P{X < \frac{k}{2} - c} \label{eq:moenster-prob}
\end{align}
Da hvis vi gør $c$-værdien meget stor, så betyder det at vi har meget få $-1$'ere, dvs. $\mathbf A_j \mathbf b$ bliver stor.\\

Nu benytter vi \cref{eq:chernoff-2}, hvor der skal gælde $0 < \delta \leq 1$:
\begin{align}
  \P{X < (1 - \delta) \mu } < e^{- \frac{\delta^2 \mu}{2} } = e^{- \frac{\delta^2 k}{4} } = \frac{1}{n^2} \label{eq:set-balance}
\end{align}

Når vi vælger $\delta^2 = \frac{8 \ln n}{k} \Longrightarrow \delta = 2 \sqrt{\frac{2 \ln n}{k}}$.\\

Skriver vi videre på venstresiden af \cref{eq:set-balance} får vi:
\begin{align}
  \P{X < (1 - \delta) \mu }
  &= \P{X < \frac{k}{2} - \delta \mu } \label{eq:indsaat-vaardi-for-mu} \\
  &= \P{\mathbf A_j \mathbf b > 2 \delta \mu} \label{eq:brug-tidligere-eq} \\
  &= \P{\mathbf A_j \mathbf b > k \delta} \nonumber \\
  &= \P{\mathbf A_j \mathbf b > 2 \sqrt{2 k \ln n}} \label{eq:brug-vaardi-for-delta} \\
  &\geq \P{\mathbf A_j \mathbf b > 2 \sqrt{2 n \ln n}} \label{eq:saat-k-til-n}
\end{align}

I \cref{eq:indsaat-vaardi-for-mu} ganger vi parentesen ud og indsætter vores værdi for $\mu$ på leddet uden $\delta$.\\
I \cref{eq:brug-tidligere-eq} benytter vi hvad vi fandt frem til i \cref{eq:moenster-prob}.\\
I \cref{eq:brug-vaardi-for-delta} indsætter vi vores værdier for $k$ og $\delta$.\\
I \cref{eq:saat-k-til-n} benytter vi, at der ikke kan være flere 1-taller end længden af vektor $\mathbf b$, altså $k \leq n$. Herved bliver uligheden sværere at opfylde, så sandsynligheden falder.\\


Hermed har vi altså vist, at for en række $\mathbf A_j$ har vi følgende, hvor medfører pilen gælder ved symmetri:
$$
\P{\mathbf A_j \mathbf b > 2 \sqrt{2 n \ln n}} < \frac{1}{n^2}
\Longleftrightarrow
\P{\mathbf A_j \mathbf b < - 2 \sqrt{2 n \ln n}} < \frac{1}{n^2}
$$

Vi kan nu bruge union bound til at få:
$$
\P{|\mathbf A_j \mathbf b| > 2 \sqrt{2 n \ln n}} < \frac{2}{n^2}
$$

Går vi nu videre fra bare at kigge på en række $\mathbf A_i$ til hele matrix $\mathbf A$ får vi, igen ved union bound, at:
\begin{align*}
  \P{||\mathbf{Ab}||_\infty > 2 \sqrt{2 n \ln n}}
  = \P{ \bigcup_{j=1}^n  \curly{|\mathbf A_j \mathbf b| > 2 \sqrt{2 n \ln n}} }
  \leq n * \frac{2}{n^2}
  = \frac{2}{n}
\end{align*}



Altså har vi nu vist, at med sandsynlighed $\geq 1 - \frac{2}{n}$ gælder $||\mathbf{Ab}||_\infty \leq 2 \sqrt{2 n \ln n}$.









\end{document}
