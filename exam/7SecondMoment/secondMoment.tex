\documentclass[a4paper,11pt]{article}
% Various packages
\usepackage{siunitx}
\usepackage[utf8]{inputenc} % æøå
\usepackage[T1]{fontenc} % mere æøå
\usepackage[danish]{babel} % orddeling
\usepackage{verbatim} % så man kan skrive ren tekst
\usepackage{graphicx}
\graphicspath{{assets/}}
\usepackage{a4wide}
\usepackage{url}
\usepackage[left=2cm,top=2cm,bottom=1.5cm,right=2cm]{geometry}
\usepackage{amsmath}
\usepackage{amssymb}
\usepackage{amsthm}
\usepackage{wrapfig}
\usepackage{fixme}
\usepackage{color}
\usepackage[makeroom]{cancel}
\usepackage{pstricks}
\usepackage{pdfpages} % include pdf
\usepackage{forest}
\usepackage{float} % Use [H] in figures
\usepackage{subcaption} % For subfigures
\usepackage{color} % May be necessary if you want to color links
%\usepackage{xcolor} \pagecolor[rgb]{0,0,0} \color[rgb]{1,1,1}
\usepackage{nameref}
\usepackage{hyperref} % Make references clickable
\usepackage[nameinlink,capitalize]{cleveref} % Make eq:refs be in style (1)
\usepackage[linesnumbered, commentsnumbered, lined, ruled, vlined,
%noend  % Have no ⌊-like symbol to indicate end of scope in pseudocode
]{algorithm2e} % Doc: https://goo.gl/6bC1qZ

\crefname{equation}{}{Equations}

% Ændr på navnene der vises når man bruger \autoref{label}
\def\sectionautorefname{Sektion}
\renewcommand{\equationautorefname}{Ligning}
\def\figureautorefname{Figur}
\AtBeginDocument{\renewcommand{\ref}[1]{\autoref{#1}}}

% Sæt \ref{} til at kalde \autoref{}
\AtBeginDocument{\renewcommand{\ref}[1]{\autoref{#1}}}

% Ændr ''*'' i math-felter til \cdot
\DeclareMathSymbol{*}{\mathbin}{symbols}{"01}

% Sæt farver for interne referencer og links
\definecolor{darkblue}{RGB}{25,25,112}
\hypersetup{
	colorlinks=true,    %set true if you want colored links
	linktoc=all,        %set to all if you want both sections and subsections linked
	linkcolor=darkblue, %choose some color if you want links to stand out
	filecolor=blue,     %
	citecolor=black,    %
	urlcolor=cyan,      %
}

% Set indentation to 0:
\setlength\parindent{0pt}

% Keywords relateret til algorithm2e pakken
\newcommand{\True}{\textbf{true}}\newcommand{\False}{\textbf{false}}
\SetStartEndCondition{ }{}{}%
\SetKwProg{Fn}{def}{\string:}{}
\SetKw{KwTo}{to}
\SetKwFor{For}{for}{}{}%
\SetKwFor{ForEach}{foreach}{}{}%
\SetKwIF{If}{ElseIf}{Else}{if}{}{elif}{else}{end}%
\SetKwFor{While}{while}{}{end}\SetKwProg{Fn}{}{}{}
\SetKwInOut{Input}{input}\SetKwInOut{Output}{output}
\setlength{\algomargin}{3em}\DontPrintSemicolon

\newcommand{\longspace}{{\ \ \ \ \ \ \ \ \ \ \ \ \ \ }}
% \renewcommand{\P}{{\mathbb P}}
\newcommand{\R}{{\mathbb R}}
\newcommand{\E}{{\mathbb E}}
\newcommand{\event}{{\mathcal{E}}}
\newcommand{\parfrac}[1]{\frac{\partial}{\partial #1}}
\renewcommand{\num}{{\textrm{num} }}
\newcommand{\size}{{\textrm{size} }}
\newcommand{\ift}{{\textrm{if } }}

\newcommand{\pfrac}[2]{\left( \frac{#1}{#2} \right)}

% Dynamiske (), <>, ceil, floor
\newcommand{\p}[1]{\left( #1 \right)}
\newcommand{\pbig}[1]{\big( #1 \big)}
\newcommand{\pBig}[1]{\Big( #1 \Big)}
\newcommand{\pbigg}[1]{\bigg( #1 \bigg)}
\newcommand{\curly}[1]{\left\{ #1 \right\}}
\renewcommand{\square}[1]{\left[ #1 \right]}
\newcommand{\larr}[1]{\left< #1 \right>}
\newcommand{\ceil}[1]{\left\lceil #1 \right\rceil}
\newcommand{\floor}[1]{\left\lfloor #1 \right\rfloor}

\renewcommand{\P}[1]{\mathbb{P} \square{ #1 } }

\author{Søren Mulvad, rbn601}

\title{Eksamensdisposition - Second Moment}

\begin{document}
\maketitle

\begin{itemize}
  \item \textbf{Problem}
  \item \textbf{Kvalitet af Count Sketch estimatet}
  \begin{itemize}
    \item Forventet værdi
    \item Varians
    \item Afvigelse fra korrekt resultat
  \end{itemize}
  \item \textbf{Median trick}
\end{itemize}




%%%%%%%%%%%%%%%%%%%%%%%%%%%%%%%%%%%%%%%%%%%%%%%%%%%%%%%%%%%
%%%%%%%%%%%%%%%%%%%%%%%%%%%%%%%%%%%%%%%%%%%%%%%%%%%%%%%%%%%
%%%%%%%%%%%%%%%%%%%%%%%%%%%%%%%%%%%%%%%%%%%%%%%%%%%%%%%%%%%
\newpage
%%%%%%%%%%%%%%%%%%%%%%%%%%%%%%%%%%%%%%%%%%%%%%%%%%%%%%%%%%%
%%%%%%%%%%%%%%%%%%%%%%%%%%%%%%%%%%%%%%%%%%%%%%%%%%%%%%%%%%%
%%%%%%%%%%%%%%%%%%%%%%%%%%%%%%%%%%%%%%%%%%%%%%%%%%%%%%%%%%%
\section{Eksamensdisposition - Second Moment}

\subsection{Problem}

Vi har en strøm af par $(j_0, \Delta_0), \dots, (j_{s-1}, \Delta_{s-1}) \in [n]\times \mathbb Z$ (sættet af alle heltal).

Vi definerer frekvensen $f_j$ for hver eneste $j \in [n]$ som
$$
f_j = \summ{i \in [s], j_i = j} \, \Delta_i
$$

Derudover definerer vi det $m$'te moment $F_m = \sum_{j \in [n]} f_j^m = ||f||_m^m$. Vi ønsker nu at estimere det 2. moment $F_2 = \sum_{i \in [n]} f_i^2$ ved kun at bruge $k$ tællere.



\subsection{Count Sketch Algoritme}

\begin{algorithm}[H] \caption{Count Sketch} \label{alg:bcs}
  \nonl Init\;
  $k = \ceil{\frac{8}{\epsilon^2}}$\;
  $C[0, \dots, k-1] = 0$\;
  $h =$ 4-universel $h : [n] \rightarrow [k]$\;
  $s =$ 4-universel $s : [n] \rightarrow \{-1, +1 \}$\;
  \nonl Process $(j, \Delta)$\;
  $C[h(j)]$ += $s(j) * \Delta$\;
  \nonl Output\;
  \Return $X = \sum_{j \in [k]} C[j]^2$
\end{algorithm}\vspace{1em}

Hvor Process $(j, \Delta)$ svarer til vi løbende kører alle par i strømmen igennem i linje 4, hvor vi potentielt kunne stoppe på et vilkårligt tidspunkt.

\subsection{Kvalitet af Count Sketch estimatet}

\subsubsection{Forventet værdi}

Vi har $C[b] = \sum_{j \in [n]} s(j) f_j [h(j) = b]$, så:
\begin{align*}
  X
  &= \sum_{b \in [k]} \p{ \sum_{j \in [n]} s(j) f_j [h_j = b] }^2\\
  &= \sum_{b \in [k]} \sum_{i,j \in [n]} s(i) s(j) f_i f_j [h(i) = b = h(j)]\\
  &= \sum_{i,j \in [n]} s(i)s(j) f_i f_j [h(i) = h(j)]\\
  &= \sum_{i \in [n]} f_i^2 + \summ{i,j \in [n]\\ i \neq j} s(i) s(j) f_i f_j [h(i) = h(j)]\\
  &= F_2 + Y
\end{align*}

Hvis vi nu kan vise $\E{Y} = 0$ har vi da vist $\E{X} = F_2$. Pga. vi sagde vores hashing-funktion er 4-universel er $s(i), s(j), h(i)$ og $h(j)$ uafhængige. Derudover har vi $\E{s(i)} = 0$, så alle led i summen bliver 0 hvorved $\E{Y} = 0$.

\subsection{Varians}
Vi ønsker at bestemme variansen af $X$ nu:

\begin{align}
  \Var{X}
  &= \Var{Y} = \E{Y^2} - \underbrace{\E{Y}^2}_{0} \nonumber \\
  &= \E{\p{ \ \summ{i, j \in [n]\\ i \neq j} s(i) s(j) f_i f_j [h(i) = h(j)]  }^2} \nonumber \\
  &= \summ{i, j, i', j' \in [n]\\ i \neq j, \ i' \neq j'} \E{ \pBig{s(i) s(j) f_i f_j [h(i) = h(j)]} \pBig{s(i') s(j') f_{i'} f_{j'} [h(i') = h(j')]  } } \label{eq:monster-sum}
\end{align}

Nu tager vi udgangspunkt i ét af leddene i summen i \cref{eq:monster-sum}. Hvis vi har den situation at en af nøglerne er unik, f.eks. $i \notin \{j, i', j'\}$, så vil $i$ pr. 4-universaliteten være uafhængig af $j, i', j'$ samt hashfunktionen $h$. hvorved vi ville kunne sætte $\E{s(i)}$ uden for en parentes i \cref{eq:monster-sum} op. Da $\E{s(i)} = 0$ vil vi derfor få at et led i summen i \cref{eq:monster-sum} med en unik nøgle vil give $0$.\\

Vi kan derfor begrænse vores opmærksomhed til kun de led der ikke har unikke nøgler. Siden $i \neq j$ og $i' \neq j'$ må vi enten have $(i, j) = (i', j')$ eller $(i, j) = (j', i')$, altså 2 cases for hvert $i$ og $j$. Derfor:
\begin{align}
  \text{Eq. \cref{eq:monster-sum}}
  &= 2 \summ{i, j \in [n]\\ i \neq j} \E{\pBig{s(i) s(j) f_i f_j [h(i) = h(j)]}^2} \label{eq:continue-calcing} \\
  &= 2 \summ{i, j \in [n]\\ i \neq j} \E{f_i^2 f_j^2 [h(i) = h(j)]} \label{eq:simplify-square} \\
  &= 2 \summ{i, j \in [n]\\ i \neq j} (f_i^2 f_j^2) \P{h(i) = h(j)} \label{eq:exp-to-prob} \\
  &= 2 \summ{i, j \in [n]\\ i \neq j} \frac{f_i^2 f_j^2}{k} \nonumber \\
  &= \frac{2}{k} \ \summ{i, j \in [n]\\ i \neq j} f_i^2 f_j^2 \nonumber \\
  &< \frac{2}{k} \pBig{\sum_{i \in [n]} f_i^2}^2 \label{eq:two-vars-to-one} \\
  &= 2 F_2^2 / k \nonumber
\end{align}

I \cref{eq:continue-calcing} bruger vi, at der netop var 2 cases hvor nøglerne var lig hinanden, og så har vi bare sat $i' = i$ og $j' = j$ i summen.\\
I \cref{eq:simplify-square} bruger vi, at $s(x)^2 = 1*1$ eller $(-1)(-1) = 1$ og indikatorvariablen er 0 eller 1, så opløftet i anden er den lig sig selv.\\
I \cref{eq:exp-to-prob} benytter vi linearity of expectation og igen at $[h(i) = h(j)]$ er en indikatorvariabel.\\
I \cref{eq:two-vars-to-one} må vores udtryk være mindre da vi ser bort fra alle de led hvor $i \neq j$.

\subsubsection{Afvigelse ved brug af Chebyshev}
Da vi nu har vist
$$
  \E{X} = F_2
  \longspace
  \Var{X} < \frac{2 F_2^2}{k}
$$

medfører det jf. Chebyshev's ulighed og vores valg af $k$ at:
$$
  \P{|X - F_2| \geq \epsilon F_2 }
  \leq \frac{\Var{X}}{(\epsilon F_2)^2}
  \leq \frac{2}{(k \epsilon^2)}
  = \frac{1}{4}
$$


\subsection{Median trick}
Vi laver $t$ uafhængige estimater $X_0, \dots, X_{t-1}$ i parallel (ved at bruge forskellige hashfunktioner) og returnerer medianen $X_{(\ceil{t/2})}$ af de $t$ svar. Vi siger $X_i$ fejler hvis $|X_i - \E{X}| \geq \epsilon F_2$.\\

Lad $B_i = [X_i \text{ fejler}]$ og lad $B$ være antallet der fejler:
$$
B = \sum_{i \in [t]} B_i
$$


Da har vi, at hvis $X_{(\ceil{t/2})}$ fejler, så betyder det at $B \geq t/2$.\\
Da $\P{B_i = 1} \leq \frac{1}{4}$ må den forventede værdi af $B$ være
$$
\E{B} = \mu = \sum_{i \in [t]} \E{B_i} \leq t/4
$$
Da kan vi beregne:
$$
\P{\text{Median fejler}}
= \P{B \geq 2 \mu}
= \P{B \geq (1 + \delta) \mu }
\leq e^{- \delta^2 \mu / 3}
\leq e^{-t/12}
$$

Hvor vi bruger Chernoff Bounds til at begrænse sandsynligheden, og herudover i eksponentialet benytter $\delta = 1$.\\

Herved har vi altså væsentligt begrænset sandsynligheden for at få noget rimelig forkert.




\end{document}
