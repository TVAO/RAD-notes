\documentclass[a4paper,11pt]{article}
% Various packages
\usepackage{siunitx}
\usepackage[utf8]{inputenc} % æøå
\usepackage[T1]{fontenc} % mere æøå
\usepackage[danish]{babel} % orddeling
\usepackage{verbatim} % så man kan skrive ren tekst
\usepackage{graphicx}
\graphicspath{{assets/}}
\usepackage{a4wide}
\usepackage{url}
\usepackage[left=2cm,top=2cm,bottom=1.5cm,right=2cm]{geometry}
\usepackage{amsmath}
\usepackage{amssymb}
\usepackage{amsthm}
\usepackage{wrapfig}
\usepackage{fixme}
\usepackage{color}
\usepackage[makeroom]{cancel}
\usepackage{pstricks}
\usepackage{pdfpages} % include pdf
\usepackage{forest}
\usepackage{float} % Use [H] in figures
\usepackage{subcaption} % For subfigures
\usepackage{color} % May be necessary if you want to color links
%\usepackage{xcolor} \pagecolor[rgb]{0,0,0} \color[rgb]{1,1,1}
\usepackage{nameref}
\usepackage{hyperref} % Make references clickable
\usepackage[nameinlink,capitalize]{cleveref} % Make eq:refs be in style (1)
\usepackage[linesnumbered, commentsnumbered, lined, ruled, vlined,
%noend  % Have no ⌊-like symbol to indicate end of scope in pseudocode
]{algorithm2e} % Doc: https://goo.gl/6bC1qZ

\crefname{equation}{}{Equations}

% Ændr på navnene der vises når man bruger \autoref{label}
\def\sectionautorefname{Sektion}
\renewcommand{\equationautorefname}{Ligning}
\def\figureautorefname{Figur}
\AtBeginDocument{\renewcommand{\ref}[1]{\autoref{#1}}}

% Sæt \ref{} til at kalde \autoref{}
\AtBeginDocument{\renewcommand{\ref}[1]{\autoref{#1}}}

% Ændr ''*'' i math-felter til \cdot
\DeclareMathSymbol{*}{\mathbin}{symbols}{"01}

% Sæt farver for interne referencer og links
\definecolor{darkblue}{RGB}{25,25,112}
\hypersetup{
	colorlinks=true,    %set true if you want colored links
	linktoc=all,        %set to all if you want both sections and subsections linked
	linkcolor=darkblue, %choose some color if you want links to stand out
	filecolor=blue,     %
	citecolor=black,    %
	urlcolor=cyan,      %
}

% Set indentation to 0:
\setlength\parindent{0pt}

% Keywords relateret til algorithm2e pakken
\newcommand{\True}{\textbf{true}}\newcommand{\False}{\textbf{false}}
\SetStartEndCondition{ }{}{}%
\SetKwProg{Fn}{def}{\string:}{}
\SetKw{KwTo}{to}
\SetKwFor{For}{for}{}{}%
\SetKwFor{ForEach}{foreach}{}{}%
\SetKwIF{If}{ElseIf}{Else}{if}{}{elif}{else}{end}%
\SetKwFor{While}{while}{}{end}\SetKwProg{Fn}{}{}{}
\SetKwInOut{Input}{input}\SetKwInOut{Output}{output}
\setlength{\algomargin}{3em}\DontPrintSemicolon

\newcommand{\longspace}{{\ \ \ \ \ \ \ \ \ \ \ \ \ \ }}
% \renewcommand{\P}{{\mathbb P}}
\newcommand{\R}{{\mathbb R}}
\newcommand{\E}{{\mathbb E}}
\newcommand{\event}{{\mathcal{E}}}
\newcommand{\parfrac}[1]{\frac{\partial}{\partial #1}}
\renewcommand{\num}{{\textrm{num} }}
\newcommand{\size}{{\textrm{size} }}
\newcommand{\ift}{{\textrm{if } }}

\newcommand{\pfrac}[2]{\left( \frac{#1}{#2} \right)}

% Dynamiske (), <>, ceil, floor
\newcommand{\p}[1]{\left( #1 \right)}
\newcommand{\pbig}[1]{\big( #1 \big)}
\newcommand{\pBig}[1]{\Big( #1 \Big)}
\newcommand{\pbigg}[1]{\bigg( #1 \bigg)}
\newcommand{\curly}[1]{\left\{ #1 \right\}}
\renewcommand{\square}[1]{\left[ #1 \right]}
\newcommand{\larr}[1]{\left< #1 \right>}
\newcommand{\ceil}[1]{\left\lceil #1 \right\rceil}
\newcommand{\floor}[1]{\left\lfloor #1 \right\rfloor}

\renewcommand{\P}[1]{\mathbb{P} \square{ #1 } }

\author{Søren Mulvad, rbn601}

\title{Eksamensdisposition - Distinct Elements}

\begin{document}
\maketitle

\begin{itemize}
  \item \textbf{Problem}
  \item \textbf{AMS algoritmen}
  \item \textbf{Analyse af sandsynlighed for fejlafvigelse}
  \begin{itemize}
    \item For højt estimat $\hat d$
    \item For lavt estimat $\hat d$
  \end{itemize}
  \item \textbf{Median tricket}
\end{itemize}


%%%%%%%%%%%%%%%%%%%%%%%%%%%%%%%%%%%%%%%%%%%%%%%%%%%%%%%%%%%
%%%%%%%%%%%%%%%%%%%%%%%%%%%%%%%%%%%%%%%%%%%%%%%%%%%%%%%%%%%
%%%%%%%%%%%%%%%%%%%%%%%%%%%%%%%%%%%%%%%%%%%%%%%%%%%%%%%%%%%
\newpage
%%%%%%%%%%%%%%%%%%%%%%%%%%%%%%%%%%%%%%%%%%%%%%%%%%%%%%%%%%%
%%%%%%%%%%%%%%%%%%%%%%%%%%%%%%%%%%%%%%%%%%%%%%%%%%%%%%%%%%%
%%%%%%%%%%%%%%%%%%%%%%%%%%%%%%%%%%%%%%%%%%%%%%%%%%%%%%%%%%%
\section{Eksamensdisposition - Distinct Elements}

% Sektion 2 og 3
% Bemærk at det de kalder "2-universal" er det som vi kalder "strong universal" eller "2independent." I section 2 gennemgik vi ikke brugen af funktionen g, men nøjedes med at gemme elementerne fra strømmen direkte i B.


\subsection{Problem}
Vi lader en strøm $j_1, \dots, j_s \in [n]$ passere. Da lader vi $f_j$ betegne antal forekomster af $j$ i strømmen, svarende til antal elementer $\{i | j_i = j \}$.\\

Nu ønsker vi at give et estimat på $d$, som beskriver hvor mange forskellige elementer der var i strømmen, hvor
$$
S = \{j | f_j > 0 \}, \longspace d = |S|
$$

\subsection{AMS algoritmen}

Vi definerer nu for et heltal $y > 0$ funktionen zeros$(y)$ til at være antal trailing 0'er i bit-formen af $y$, som formelt er
$$
\text{zeros}(y) = \max\{i : y \bmod 2^i = 0\}
$$
F.eks. vil $y = 288_{10}$ give $\text{zeros}(y) = 1001 \underbrace{00000}_{zeros} = 5$ og vi ser $y \bmod 2^5 = 0$ mens $y \bmod 2^6 \neq 0$.\\


\begin{algorithm}[H] \caption{AMS} \label{alg:ams}
  \SetKwFunction{zeros}{zeros}%
	%\BlankLine
  \nonl Init\;
  $h =$ new random strong-universel hash function s.t. $h : [n] \rightarrow [n]$\;
  $z = 0$\;
  \nonl Process $j$\;
  \If{$\zeros{h(j)} > z$}{
    $z = \zeros{h(j)}$\;
  }
  \Return $\hat d = 2^{z + 1/2}$
\end{algorithm}\vspace{1em}

Hvor Process $j$ svarer til vi løbende kører alle $j$'er i strømmen igennem i linje 3, hvor vi potentielt kunne stoppe på et vilkårligt tidspunkt.\\

Intuitionen for algoritmen er, at vi forventer 1 ud af de $d$ unikke elementer hashes så \texttt{zeros}$(h(j)) \geq \log d$, og vi forventer ikke nogle elementer at ramme \texttt{zeros}$(h(j)) \gg \log d$. Derfor vil maksværdien af \texttt{zeros}$(h(j))$ være en god approksimation af $\log d$.


\subsection{Analyse af sandsynlighed for fejlafvigelse}

For ethvert $j \in [n]$ og ethvert heltal $r \geq 0$, definer da begivenheden:\\
$\event_{r,j}$: \texttt{zeros}$(h(j)) \geq r$\\

Lad nu $X_{r,j}$ være en indikatorvariabel som er 1 hvis $\event_{r,j}$ er sand, 0 ellers. Lad $Y_r$ være en stokastisk variabel for et givent $r$ som beskriver antal elementer $j$ der både indgår i strømmen og opfylder $\event_{r,j}$:
\begin{align}
  Y_r = \sum_{j: f_j > 0} X_{r,j} \label{eq:yr-def}
\end{align}

Lad $t$ beskrive værdien af $z$ idet algoritmen terminerer. Så har vi følgende:
\begin{align}
  t \geq r \quad \Longleftrightarrow \quad Y_r > 0  \label{eq:t-greater-eq-r}
\end{align}
Idet der minimum har været et element der opfyldte $\event_{r,j}$.\\

Vi kan også omskrive det til, at såfremt intet element opfyldte $\event_{r,j}$ har vi:
\begin{align}
  t \leq r - 1 \quad \Longleftrightarrow \quad Y_r = 0 \label{eq:t-leq-r}
\end{align}

Siden $h(j)$ er uniformt distribueret over $(\lg n)$-bitstrengene har vi:
$$
\E{X_{r,j}} = \P{\mathtt{zeros}(h(j)) \geq r} = \P{h(j) \bmod 2^r = 0} = \frac{1}{2^r}
$$

Vi kan nu bestemme forventning $Y_r$ som følger, idet vi bruger linearity of expectation på \cref{eq:yr-def} og vi har defineret $d$ til at være antal unikke elementer:
\begin{align}
  \E{Y_r} = \sum_{j: f_j > 0} \E{X_{r,j}} = \frac{d}{2^r}
\end{align}


Variansen fås til:
\begin{align}
  \Var{Y_r}
  &= \sum_{j: f_j > 0} \Var{X_{r,j}} \label{eq:parvis-uaf} \\
  &= \sum_{j: f_j > 0} \E{X_{r,j}^2} - \E{X_{r,j}}^2 \label{eq:var-def} \\
  &\leq \sum_{j: f_j > 0} \E{X_{r,j}^2} \nonumber \\
  &= \sum_{j: f_j > 0} \E{X_{r,j}} \label{eq:indi-var} \\
  &= \frac{d}{2^r} \nonumber
\end{align}

Hvor vi i \cref{eq:parvis-uaf} bruger den parvise uafhængighed der følger af at vi valgte en stærk universel hashing-funktion, i \cref{eq:var-def} bruger definitionen på varians og i \cref{eq:indi-var} bruger at de er indikator-variabler.\\

Lad os nu definere $\hat d$ til at være estimatet af $d$ som algoritmen returnerer, hvorved vi har $\hat d = 2^{t + \frac{1}{2}}$.

Vi lader nu $a$ være det mindste heltal så $2^{a + \frac{1}{2}} \geq 6d$. Så får vi:
\begin{align}
  \P{\hat d \geq 6d}
  &= \P{t \geq a} \label{eq:t-stoerre-a} \\
  &= \P{Y_a > 0} \label{eq:benyt-ulighed} \\
  &= \P{Y_a \geq 1} \nonumber \\
  &= \frac{\E{Y_r}}{1} \label{eq:benyt-markov} \\
  &\leq \frac{d}{2^a} < \frac{1}{4} \label{eq:benyt-forventet-yr}
\end{align}

I \cref{eq:t-stoerre-a} benytter vi værdien for $\hat d$ og vores definition på $a$.\\
I \cref{eq:benyt-ulighed} benytter vi $t \geq r \Longleftrightarrow Y_r > 0$ fra \cref{eq:t-greater-eq-r}.\\
I \cref{eq:benyt-markov} bruger vi Markovs ulighed.\\
I \cref{eq:benyt-forventet-yr} benytter vi vores værdi for $\E{Y_r}$. Man får herefter en talværdi mindre end $1/4$ såfremt man indsatte $a$.\\\\

Lad os tilsvarende kigge på sandsynligheden for at vi får noget for småt. Lad $b$ være det største heltal så $2^{b + \frac{1}{2}} \leq d/6$. Da får vi:
\begin{align}
  \P{\hat d \leq d/6}
  &= \P{t \leq b} \label{eq:benyt-d-og-b} \\
  &= \P{Y_{b+1} = 0}
  = \P{ |Y_{b+1} - \E{Y_{b+1}}| \geq \E{Y_{b+1}} } \label{eq:benyt-t-mindre-b} \\
  &\leq \frac{\Var{Y_{b+1}}}{ \E{Y_{b+1}}^2 }
  = \frac{1}{\E{Y_{b+1}}}
  \leq \frac{2^{b+1}}{d} \label{eq:benyt-chebyshev}
  < \frac{1}{4}
\end{align}

I \cref{eq:benyt-d-og-b} benytter vi vores værdier for $\hat d$ og $b$.\\
I \cref{eq:benyt-t-mindre-b} benytter vi $t \leq r \Longleftrightarrow Y_b = 0$ fra \cref{eq:t-leq-r}.\\
I \cref{eq:benyt-chebyshev} benytter vi Chebyshevs ulighed og at $\Var{Y_r} = \E{Y_r}$.\\

Vi ser at garantierne er relativt små. Det ses både ved at $\hat d$ kan afvige med en del og at vores sandsynligheder kun er begrænset af omkring 25 \%.

\subsection{Median tricket}
Lav nu $k$ uafhængige estimater $\hat d_0, \dots, \hat d_{k-1}$ i parallel (ved at bruge forskellige hashfunktioner) og returner medianen $\hat d_{(\ceil{k/2})}$ af de $k$ svar.\\


Lad $Z_i = [\hat d_i \geq t]$ og definer $Z$ til
$$
Z = \sum_{i \in [k]} Z_i
$$

Hvis vi bruger $t = 6d$ får vi:
$$
  \E{Z} = \sum_{i \in [k]} \P{Z_i} \leq k * 1/4 = k/4
$$

Da får vi
\begin{align}
  \P{\hat d_{(\ceil{k/2})} > t}
  = \P{Z > \ceil{k/2}}
  = \P{Z \geq 2 \E{Z}}
  \leq e^{- \mu / 3}
  = e^{- k / 12}
\end{align}

Herved har vi altså væsentligt begrænset sandsynligheden for at få noget rimelig forkert.


% Gad ikke lære BJKST-algoritmen og beviserne hertil udenad, men var småt begyndt. Udkommenteret for nu.

% \subsection{BJKST algoritmen}
% Samme setup som før med en strøm og problem vi ønsker at løse. Nu har vi følgende algoritme:
% \begin{algorithm}[H] \caption{BJKST} \label{alg:bjkst}
%   \SetKwFunction{zeros}{zeros}%
% 	%\BlankLine
%   \nonl Init\;
%   $h =$ new random strong-universel hash function s.t. $h : [n] \rightarrow [n]$\;
%   $z = 0$\;
%   $B = \emptyset$\;
%   \nonl Process $j$\;
%   \If{$\zeros{h(j)} \geq z$}{
%     $B = B \cup \{ j \}$\;
%   }
%   \While{$|B| > k$}{
%     remove all $i \in B$ where $\zeros{h(i)} = z$\;
%     $z += 1$\;
%   }
%   \nonl Output\;
%   \Return $\hat d = |B|2^z$
% \end{algorithm}\vspace{1em}

% Vi ser at $B$ er sættet af alle $j$ hvor \texttt{zeros}$(h(j)) \geq z$. Ud af de $d$ unikke elementer i strømmen vil vi forvente at $d/2^z$ falder ind i dette sæt. Derfor vil vores return-værdi være et fint estimat.\\

% Definer nu $k = c / \epsilon^2$, hvor $c$ vælges senere.


% Lad os definere at en fejl-begivenhed sker når $\hat d$ ikke er en $(1 + \pm \epsilon)$-approksimation af $d$:
% \begin{align}
%   \mathtt{FAIL}
%   \Longleftrightarrow
%   | \hat d - d | \geq \epsilon d
%   \Longleftrightarrow
%   | Y_t 2^t - d | \geq \epsilon d
%   \Longleftrightarrow
%   \left| Y_t - \frac{d}{2^t}   \right| \geq \frac{\epsilon d}{2^t}
% \end{align}


\end{document}
