\documentclass[a4paper,11pt]{article}
% Various packages
\usepackage{siunitx}
\usepackage[utf8]{inputenc} % æøå
\usepackage[T1]{fontenc} % mere æøå
\usepackage[danish]{babel} % orddeling
\usepackage{verbatim} % så man kan skrive ren tekst
\usepackage{graphicx}
\graphicspath{{assets/}}
\usepackage{a4wide}
\usepackage{url}
\usepackage[left=2cm,top=2cm,bottom=1.5cm,right=2cm]{geometry}
\usepackage{amsmath}
\usepackage{amssymb}
\usepackage{amsthm}
\usepackage{wrapfig}
\usepackage{fixme}
\usepackage{color}
\usepackage[makeroom]{cancel}
\usepackage{pstricks}
\usepackage{pdfpages} % include pdf
\usepackage{forest}
\usepackage{float} % Use [H] in figures
\usepackage{subcaption} % For subfigures
\usepackage{color} % May be necessary if you want to color links
%\usepackage{xcolor} \pagecolor[rgb]{0,0,0} \color[rgb]{1,1,1}
\usepackage{nameref}
\usepackage{hyperref} % Make references clickable
\usepackage[nameinlink,capitalize]{cleveref} % Make eq:refs be in style (1)
\usepackage[linesnumbered, commentsnumbered, lined, ruled, vlined,
%noend  % Have no ⌊-like symbol to indicate end of scope in pseudocode
]{algorithm2e} % Doc: https://goo.gl/6bC1qZ

\crefname{equation}{}{Equations}

% Ændr på navnene der vises når man bruger \autoref{label}
\def\sectionautorefname{Sektion}
\renewcommand{\equationautorefname}{Ligning}
\def\figureautorefname{Figur}
\AtBeginDocument{\renewcommand{\ref}[1]{\autoref{#1}}}

% Sæt \ref{} til at kalde \autoref{}
\AtBeginDocument{\renewcommand{\ref}[1]{\autoref{#1}}}

% Ændr ''*'' i math-felter til \cdot
\DeclareMathSymbol{*}{\mathbin}{symbols}{"01}

% Sæt farver for interne referencer og links
\definecolor{darkblue}{RGB}{25,25,112}
\hypersetup{
	colorlinks=true,    %set true if you want colored links
	linktoc=all,        %set to all if you want both sections and subsections linked
	linkcolor=darkblue, %choose some color if you want links to stand out
	filecolor=blue,     %
	citecolor=black,    %
	urlcolor=cyan,      %
}

% Set indentation to 0:
\setlength\parindent{0pt}

% Keywords relateret til algorithm2e pakken
\newcommand{\True}{\textbf{true}}\newcommand{\False}{\textbf{false}}
\SetStartEndCondition{ }{}{}%
\SetKwProg{Fn}{def}{\string:}{}
\SetKw{KwTo}{to}
\SetKwFor{For}{for}{}{}%
\SetKwFor{ForEach}{foreach}{}{}%
\SetKwIF{If}{ElseIf}{Else}{if}{}{elif}{else}{end}%
\SetKwFor{While}{while}{}{end}\SetKwProg{Fn}{}{}{}
\SetKwInOut{Input}{input}\SetKwInOut{Output}{output}
\setlength{\algomargin}{3em}\DontPrintSemicolon

\newcommand{\longspace}{{\ \ \ \ \ \ \ \ \ \ \ \ \ \ }}
% \renewcommand{\P}{{\mathbb P}}
\newcommand{\R}{{\mathbb R}}
\newcommand{\E}{{\mathbb E}}
\newcommand{\event}{{\mathcal{E}}}
\newcommand{\parfrac}[1]{\frac{\partial}{\partial #1}}
\renewcommand{\num}{{\textrm{num} }}
\newcommand{\size}{{\textrm{size} }}
\newcommand{\ift}{{\textrm{if } }}

\newcommand{\pfrac}[2]{\left( \frac{#1}{#2} \right)}

% Dynamiske (), <>, ceil, floor
\newcommand{\p}[1]{\left( #1 \right)}
\newcommand{\pbig}[1]{\big( #1 \big)}
\newcommand{\pBig}[1]{\Big( #1 \Big)}
\newcommand{\pbigg}[1]{\bigg( #1 \bigg)}
\newcommand{\curly}[1]{\left\{ #1 \right\}}
\renewcommand{\square}[1]{\left[ #1 \right]}
\newcommand{\larr}[1]{\left< #1 \right>}
\newcommand{\ceil}[1]{\left\lceil #1 \right\rceil}
\newcommand{\floor}[1]{\left\lfloor #1 \right\rfloor}

\renewcommand{\P}[1]{\mathbb{P} \square{ #1 } }

\author{Søren Mulvad, rbn601}

\title{Eksamensdisposition - Hash tables}

\begin{document}
\maketitle

\begin{itemize}
  \item \textbf{Redegørelse for datastruktur og anvendelse}
  \item \textbf{Bestemmelse af forventet køretid for indsættelse/sletning}
  \item \textbf{Opbygning af hash table der altid sikrer konstant opslagstid}
  \item \textbf{Bestemmelse af pladsforbrug for 2-level hashing}
\end{itemize}


%%%%%%%%%%%%%%%%%%%%%%%%%%%%%%%%%%%%%%%%%%%%%%%%%%%%%%%%%%%
%%%%%%%%%%%%%%%%%%%%%%%%%%%%%%%%%%%%%%%%%%%%%%%%%%%%%%%%%%%
%%%%%%%%%%%%%%%%%%%%%%%%%%%%%%%%%%%%%%%%%%%%%%%%%%%%%%%%%%%
\newpage
%%%%%%%%%%%%%%%%%%%%%%%%%%%%%%%%%%%%%%%%%%%%%%%%%%%%%%%%%%%
%%%%%%%%%%%%%%%%%%%%%%%%%%%%%%%%%%%%%%%%%%%%%%%%%%%%%%%%%%%
%%%%%%%%%%%%%%%%%%%%%%%%%%%%%%%%%%%%%%%%%%%%%%%%%%%%%%%%%%%
\section{Eksamensdisposition - Hash tables}
\subsection{Redegørelse for datastruktur og anvendelse}
Vi bestemmer en hashfunktion $h$ der mapper et univers $U$ til sættet af tal $[m]$:
$$
  h : U \rightarrow [m] = \{0, \dots, m-1 \}
$$

For en $c$-universel hashing gælder, at for $x, y \in U$ hvor $x \neq y$, da:
$$
\P{h(x) = h(y)} \leq \frac{c}{m}
$$

Hashing funktioner benyttes særligt ofte i det man typisk kalder dictionaries. Lad os sige vi har en liste $S$ med $n = |S|$ nøgle/værdi-par. Så laver et array $L$ der er $m \geq n$ langt med lister og laver en hashing funktion $h : U \rightarrow [m]$.

Så vil $L[i]$ bestå af alle de elementer hvis nøgle hashes til dette index:
\begin{figure}[H]
  \begin{center}
  \includegraphics[width=0.6\textwidth]{dict.pdf}
  \end{center}
  \caption{Hash tabel med chaining}
  \label{fig:hash}
\end{figure}


\subsection{Bestemmelse af forventet køretid for indsættelse/sletning}

\TODO{Lav denne del}


\subsection{Opbygning af dictionary der altid sikrer konstant opslagstid}
Antag vi har en universel hash funktion $h : U \rightarrow [m]$. hvor $m \geq 1$ er et heltal vi vælger senere. For ethvert $i \in [m]$, definer $S_i = \{ x \in S | h(x) = i \}$ og $n_i = |S_i|$. For ethvert sæt $S_i$ er køretiden $O(1)$ hvis det er tomt eller kun indeholder et element. Men hvis mange elementer hasher til samme index vil det tage lang tid.\\

Lad os definere $C_h = \sum_{i \in [m]} \binom{n_i}{2}$ som er det totale antal par af nøgler i $S$ der kolliderer med hashfunktionen $h$. Vi ønsker en hashfunktion så $C_h = 0$, hvorved der for alle sæt $S_i$ gælder de er tomme eller kun indeholder ét element.\\

Vi har pr. definition af $C_h$ følgende, hvor $[h(x) = h(y)]$ svarer til en indikatorvariabel der er 1 hvis udsagnet er sandt, ellers 0:
\begin{align}
  C_h = \sum_{\mathclap{\substack{ \{x, y\} \in S\\ x \neq y}}} \ [h(x) = h(y)]
\end{align}

Da vi antager $h$ er universel har vi, at $\E{h(x) = h(y)} \leq 1/m$ når $x \neq y$:
\begin{align}
  \E{C_h}
  &= \E{ \ \ \sum_{\mathclap{\substack{ \{x, y\} \in S\\ x \neq y}}} \ h(x) = h(y)} \nonumber \\
  &= \sum_{\mathclap{\substack{ \{x, y\} \in S\\ x \neq y}}} \ \E{h(x) = h(y)} \label{eq:lin-of-expp} \\
  &\leq \binom{n}{2} * \frac{1}{m} \label{eq:n-choose-2} \\
  &= \frac{n!}{2!(n-2)!} * \frac{1}{m} \nonumber \\
  &= \frac{1}{2m} * \frac{n!}{(n-2)!} \nonumber \\
  &= \frac{n(n-1)}{2m} \label{eq:binom-to-frac}
\end{align}

I \cref{eq:lin-of-expp} bruger vi linearity of expectation.\\
I \cref{eq:n-choose-2} har vi $\binom{n}{2}$ da $n = |S|$ og vi vælger to elementer. Forventningen $1/m$ er givet jf. vores antagelse.\\
Frem til \cref{eq:binom-to-frac} er det simpel købmandsregning.\\


Nu kan vi bestemme:
\begin{align}
  \P{C_h > n(n-1)/m}
  &\leq \P{C_h \geq 2 \E{C_h}} \label{eq:1-til-2E} \\
  &\leq \frac{\E{C_h}}{2 \E{C_h}} \label{eq:benyt-markov} \\
  &= \frac{1}{2} \nonumber
\end{align}

I \cref{eq:1-til-2E} får vi uligheden i selve ligningen da uligheden i højre sandsynlighed gør den kan ske én gang oftere, mens vi i \cref{eq:benyt-markov} benytter vi Markovs ulighed.\\

Hvis vi vælger en ny hashfunktion $h$ hver gang $C_h > n(n-1)/m$ vil det forventede antal gange vi skal gøre det være højest $2$, da det er geometrisk distribueret.\\


Hvis vi i stedet vælger $m = n^2$ vil det forventede antal hashfunktioner $h$ vi skal prøve før vi får $C_h = 0$ være højest 2. Det ser vi ved at indsætte $m = n^2$. Da får vi:
$$
C_h \leq \frac{n(n-1)}{n^2} < 1
\quad\quad \rightarrow \quad\quad
C_h = 0
$$
efter forventet højest 2 forsøg. Det gælder da vi kun arbejder med heltal.\\

Med denne metode får vi altså en dictionary med 0 kollisioner, og derved $O(1)$ opslagstid, men til gengæld bruger vi $\Theta(n^2)$ plads.


\subsection{Bestemmelse af pladsforbrug for 2-level hashing}
Lad os nu vælge $m = n$. Nu kan vi finde en hash funktion $h$ så $C_h \leq n$ i forventet højest 2 forsøg (da $C_h \leq \frac{n(n-1)}{n} < n$ jf. hvad vi viste lige før).\\

Det andet niveau består af en 1-level hash tabel for hvert $S_i$ hvor $|S_i| = n_i \geq 1$. Så for hvert $i \in [n]$ hvor $n_i \geq 1$ lad $h_i : U \rightarrow [m_i]$ være en universel hashfunktion hvor vi definerer $m_i = n_i^2$. Hvis der sker nogle kollisioner prøver vi igen med en ny $h_i$.


%\newpage
%\subsection{Datastruktur og anvendelse}
Fuldt tilfældig
$$
  h: U \rightarrow [m] = \{ 0, \dots, m-1 \}
$$

Alle $|U|$ hashværdier uafhængige og uniforme. Tager $|U|$ plads.

I stedet kun element af tilfældighed. Vælg et primtal $p$ så $U \subseteq [p] = \mathbb Z_p$ samt tilfældige $a, b \in [p]$.

\begin{align*}
  h_{a, b}(x)   &= (ax + b) \bmod p\\
  h_{a, b}^m(x) &= h_{a, b} \bmod m
\end{align*}


Ved universel hashing har vi
$$
h : U \rightarrow [m]
$$

For $x, y \in U$ hvor $x \neq y$, da
$$
\P{h(x) = h(y)} \leq \frac{1}{m}
$$

Theorem: Hvis $a \in [p]_+ = \{1, \dots, p-1 \}, b \in [p]$ er uniforme og uafhængige så er $h_{a, b}(x)$ universel.

Theorem: Hvis $U = [2^w], w = 8, 16, 32, \dots$ og $m = 2^l$ og $a \in [2^w]$ er et tilfældigt ulige tal, så:
$$
h_a(x) = (a*x) >> (w - l)
$$


\subsection{Anvendelse}

Hash tabel med lister/kæder (dictionary). Lad os sige vi har en liste $S$ med $|S| = n$ nøgle/værdi-par.\\

Da kan vi gemme dem i vores dictionary med $O(n)$ plads og $O(1)$ forventet indsættelse og sletning.\\

Vi laver et array $L$ der er $m$ langt med lister. Så har vi en universel hash $h : U \rightarrow [m], m \geq n$.\\

Nu lader vi $L[i]$ være en liste over $S_i = \{ x \in S | h(x) = i \}$\\

Dvs. $x \in S \Longleftrightarrow x$ ligger i $L[h(x)]$.

\begin{figure}[H]
  \begin{center}
  \includegraphics[width=0.6\textwidth]{dict.pdf}
  \end{center}
  \caption{Hash tabel med chaining}
  \label{fig:hash}
\end{figure}

Herved får vi, at:
$$
| \, L[h(x)] \, | = \sum_{y \in S} [h(y) = h(x)]
$$

\subsection{Forventet køretid for indsættelse}
Tiden det tager at finde ud af om $x \in U$ er i $S$ er proportional med listens længde, $O(1 + | \, L[h(x)] \, |)$.\\

Antag nu, at $x \notin S$, $h$ er universel, $m \geq n$ og lad $I(y)$ være en indikator-variabel som er $1$ hvis $h(x) = h(y)$ og $0$ ellers. Da er det forventede antal elementer i $L[h(x)]$:

\begin{align}
  \E{ \, |L[h(x)]| \, }
  &= \E{ \sum_{y \in S} I(y)} \label{eq:indi} \\
  &= \sum_{y \in S} \E{I(y)} \label{eq:lin-of-exp} \\
  &= \sum_{y \in S} \P{h(y) = h(x)} \nonumber \\
  &\leq \sum_{y \in S} \frac{1}{m} \label{eq:1-over-m} \\
  &= n/m \label{eq:n-over-m} \\
  &\leq 1 \nonumber
\end{align}

I \cref{eq:indi} bruger vi, at antallet af elementer i listen $L[h(x)]$ må være alle dem som hasher til værdi, altså $h(y) = h(x)$.\\
I \cref{eq:lin-of-exp} bruger vi Linearity of Expectation.\\
I \cref{eq:1-over-m} benytter vi vores antagelse om at vores hashing function er universel, hvorved uligheden gælder.\\
I \cref{eq:n-over-m} benytter vi, at der i alt er $n$ elementer i $S$.


\subsection{c-universel hashing}

For $x, y \in U, x \neq y$, da er $\P{h(x) = h(y)} \leq \frac{c}{m}$.\\

Theorem: Hvis $a \in [p]_+ = \{1, \dots, p-1 \}$ og $b \in [p]$ er uniforme og uafhængige, så er $h_{a,b}(x)$ 1-universel.\\

Theorem: Hvis $U = [2^w], w = \{8, 16, 32, 64, \dots \}$ og $m = 2^l$ og $a \in [2^w]$ er tilfældig ulige, så er $h_a(x) = (a*x) >> (w-l)$ 2-universel.


\subsection{Konstant query-tid for mængde $S$}
Givet en mængde $S$, da ønsker vi at opnå konstant query tid. Det opnås f.eks. hvis $|S_i| \leq 1$ for alle $i$ hvilket vil sige der er 0 kollisioner for alle $x, y \in S$ hvor $x \neq y$.

Lad os nu definere $C_h$ til at betegne antal kollisioner $h(x) = h(y)$ for alle $x, y \in S$ hvor $x \neq y$. Lad os nu beregne den forventede værdi af $C_h$:

\begin{align}
  \E{C_h} &< \binom{n}{2} \frac{1}{m} = \frac{n(n-1)}{2m}
\end{align}


Da får vi:
$$
\P{C_h \geq \frac{n(n-1)}{m}} \leq \frac{\E{C_h}}{n(n-1)/m} = \frac{1}{2}
$$

Hvis $C_h$ bliver for stort prøver vi med et nyt $h$. Herved kan vi i alt forvente 2 forsøg.

Lad os prøve at udføre dette med $m = n^2$. Da:
$$
C_h < \frac{n(n-1)}{m} < 1 \Longrightarrow C_h = 0
$$



\subsection{2 level hashing - Pladsforbrug}

Først $m = n$ og får $C_h < \frac{n(n-1)}{m} = n-1$. For hvert $i$, gem $S_i$ med $m_i = n_i^2$ hvor $n_i = |S_i|$.

Vi kan da beregne pladsforbruget til:

\begin{align}
  O \p{ 1 + n +m + \sum_{i \in [m]} \p{ 1 + n_i + m_i }   }
  &= O \p{ n + \sum_{i \in [m]} \p{ 1 + n_i + n_i^2 }  }\\
  &= O \p{ n + \sum_{i \in [m]} \p{ 2 n_i + n_i ( n_i - 1) }  }\\
  &= O \p{ n + 2 \sum_{i \in [m]} \binom{n_i}{2} }\\
  &= O(n + C_h)\\
  &= O(n)
\end{align}

\subsection{Stærk $c$-universel}

For alle $x \neq y$ hvor $x, y \in U$ og $q, v \in [m]$, da er:

\begin{align}
  &\P{h(x) = q \text{ og } h(y) = v} = \frac{1}{m^2}\\
  &\Updownarrow \\
  (1) &\text{ $h(z)$ uniform i $[m]$} \Longrightarrow \P{h(x) = s} \leq \frac{c}{m}\\
  (2) &\text{ $h(x)$ og $h(y)$ er uafhængig}
\end{align}

Andet gøgl på tavlen:
$$
\P{h(x) = h(y)} \geq \sum_{q \in [m]} \P{h(x) = h(y) = q}
$$


\end{document}
