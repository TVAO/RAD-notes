\documentclass[a4paper,11pt]{article}
% Various packages
\usepackage{siunitx}
\usepackage[utf8]{inputenc} % æøå
\usepackage[T1]{fontenc} % mere æøå
\usepackage[danish]{babel} % orddeling
\usepackage{verbatim} % så man kan skrive ren tekst
\usepackage{graphicx}
\graphicspath{{assets/}}
\usepackage{a4wide}
\usepackage{url}
\usepackage[left=2cm,top=2cm,bottom=1.5cm,right=2cm]{geometry}
\usepackage{amsmath}
\usepackage{amssymb}
\usepackage{amsthm}
\usepackage{wrapfig}
\usepackage{fixme}
\usepackage{color}
\usepackage[makeroom]{cancel}
\usepackage{pstricks}
\usepackage{pdfpages} % include pdf
\usepackage{forest}
\usepackage{float} % Use [H] in figures
\usepackage{subcaption} % For subfigures
\usepackage{color} % May be necessary if you want to color links
%\usepackage{xcolor} \pagecolor[rgb]{0,0,0} \color[rgb]{1,1,1}
\usepackage{nameref}
\usepackage{hyperref} % Make references clickable
\usepackage[nameinlink,capitalize]{cleveref} % Make eq:refs be in style (1)
\usepackage[linesnumbered, commentsnumbered, lined, ruled, vlined,
%noend  % Have no ⌊-like symbol to indicate end of scope in pseudocode
]{algorithm2e} % Doc: https://goo.gl/6bC1qZ

\crefname{equation}{}{Equations}

% Ændr på navnene der vises når man bruger \autoref{label}
\def\sectionautorefname{Sektion}
\renewcommand{\equationautorefname}{Ligning}
\def\figureautorefname{Figur}
\AtBeginDocument{\renewcommand{\ref}[1]{\autoref{#1}}}

% Sæt \ref{} til at kalde \autoref{}
\AtBeginDocument{\renewcommand{\ref}[1]{\autoref{#1}}}

% Ændr ''*'' i math-felter til \cdot
\DeclareMathSymbol{*}{\mathbin}{symbols}{"01}

% Sæt farver for interne referencer og links
\definecolor{darkblue}{RGB}{25,25,112}
\hypersetup{
	colorlinks=true,    %set true if you want colored links
	linktoc=all,        %set to all if you want both sections and subsections linked
	linkcolor=darkblue, %choose some color if you want links to stand out
	filecolor=blue,     %
	citecolor=black,    %
	urlcolor=cyan,      %
}

% Set indentation to 0:
\setlength\parindent{0pt}

% Keywords relateret til algorithm2e pakken
\newcommand{\True}{\textbf{true}}\newcommand{\False}{\textbf{false}}
\SetStartEndCondition{ }{}{}%
\SetKwProg{Fn}{def}{\string:}{}
\SetKw{KwTo}{to}
\SetKwFor{For}{for}{}{}%
\SetKwFor{ForEach}{foreach}{}{}%
\SetKwIF{If}{ElseIf}{Else}{if}{}{elif}{else}{end}%
\SetKwFor{While}{while}{}{end}\SetKwProg{Fn}{}{}{}
\SetKwInOut{Input}{input}\SetKwInOut{Output}{output}
\setlength{\algomargin}{3em}\DontPrintSemicolon

\newcommand{\longspace}{{\ \ \ \ \ \ \ \ \ \ \ \ \ \ }}
% \renewcommand{\P}{{\mathbb P}}
\newcommand{\R}{{\mathbb R}}
\newcommand{\E}{{\mathbb E}}
\newcommand{\event}{{\mathcal{E}}}
\newcommand{\parfrac}[1]{\frac{\partial}{\partial #1}}
\renewcommand{\num}{{\textrm{num} }}
\newcommand{\size}{{\textrm{size} }}
\newcommand{\ift}{{\textrm{if } }}

\newcommand{\pfrac}[2]{\left( \frac{#1}{#2} \right)}

% Dynamiske (), <>, ceil, floor
\newcommand{\p}[1]{\left( #1 \right)}
\newcommand{\pbig}[1]{\big( #1 \big)}
\newcommand{\pBig}[1]{\Big( #1 \Big)}
\newcommand{\pbigg}[1]{\bigg( #1 \bigg)}
\newcommand{\curly}[1]{\left\{ #1 \right\}}
\renewcommand{\square}[1]{\left[ #1 \right]}
\newcommand{\larr}[1]{\left< #1 \right>}
\newcommand{\ceil}[1]{\left\lceil #1 \right\rceil}
\newcommand{\floor}[1]{\left\lfloor #1 \right\rfloor}

\renewcommand{\P}[1]{\mathbb{P} \square{ #1 } }

\author{Søren Mulvad, rbn601}

\title{Eksamensdisposition - Hash tables}

\begin{document}
\maketitle

\begin{itemize}
  \item \textbf{Redegørelse for datastruktur og anvendelse}
  \item \textbf{Bestemmelse af forventet opslagstid}
  \item \textbf{Opbygning af hash table der altid sikrer konstant opslagstid}
  \item \textbf{Bestemmelse af pladsforbrug for 2-level hashing}
\end{itemize}


%%%%%%%%%%%%%%%%%%%%%%%%%%%%%%%%%%%%%%%%%%%%%%%%%%%%%%%%%%%
%%%%%%%%%%%%%%%%%%%%%%%%%%%%%%%%%%%%%%%%%%%%%%%%%%%%%%%%%%%
%%%%%%%%%%%%%%%%%%%%%%%%%%%%%%%%%%%%%%%%%%%%%%%%%%%%%%%%%%%
\newpage
%%%%%%%%%%%%%%%%%%%%%%%%%%%%%%%%%%%%%%%%%%%%%%%%%%%%%%%%%%%
%%%%%%%%%%%%%%%%%%%%%%%%%%%%%%%%%%%%%%%%%%%%%%%%%%%%%%%%%%%
%%%%%%%%%%%%%%%%%%%%%%%%%%%%%%%%%%%%%%%%%%%%%%%%%%%%%%%%%%%
\section{Eksamensdisposition - Hash tables}
\subsection{Redegørelse for datastruktur og anvendelse}
Vi bestemmer en hashfunktion $h$ der mapper et univers $U$ til sættet af tal $[m]$:
$$
  h : U \rightarrow [m] = \{0, \dots, m-1 \}
$$

For en $c$-universel hashing gælder, at for $x, y \in U$ hvor $x \neq y$, da:
$$
\P{h(x) = h(y)} \leq \frac{c}{m}
$$

Hvis $c = 1$ kalder vi den blot for universel.\\

Vi kan implementere en 2-universel hashing funktion på f.eks. følgende måder:
$$
  ((ax + b) \bmod p) \bmod m
  \longspace
  (a' x) >> (w-l)
$$
hvor vi i MulShift har at $a'$ skal være ulige og $m = 2^l$.\\

Hashing funktioner benyttes særligt ofte i det man typisk kalder dictionaries. Lad os sige vi har en liste $S$ med $|S| = n$ nøgle/værdi-par. Så laver et array $L$ der er $m \geq n$ langt med lister og laver en hashing funktion $h : U \rightarrow [m]$.

Da vil $L[i]$ bestå af alle de elementer hvis nøgle hashes til dette index $i$:
\begin{figure}[H]
  \begin{center}
  \includegraphics[width=0.6\textwidth]{dict.pdf}
  \end{center}
  \caption{Hash tabel med chaining}
  \label{fig:hash}
\end{figure}

Vi ser nemt at denne struktur vil bruge $O(n + m)$ plads.

\subsection{Bestemmelse af forventet opslagstid}

Tiden det tager at finde ud af om $x \in U$ er i $S$ er proportional med listens længde, $O(1 + | \, L[h(x)] \, |)$.\\

Antag nu, at $x \notin S$, $h$ er universel, $m \geq n$ og lad $I(y)$ være en indikator-variabel som er $1$ hvis $h(x) = h(y)$ og $0$ ellers. Da er det forventede antal elementer i $L[h(x)]$:

\begin{align}
  \E{ \, |L[h(x)]| \, }
  &= \E{ \sum_{y \in S} I(y)} \label{eq:indi} \\
  &= \sum_{y \in S} \E{I(y)} \label{eq:lin-of-exp} \\
  &= \sum_{y \in S} \P{I(y) = 1} \nonumber \\
  &\leq \sum_{y \in S} \frac{1}{m} \label{eq:1-over-m} \\
  &= n/m \label{eq:n-over-m} \\
  &\leq 1 \nonumber
\end{align}

I \cref{eq:indi} bruger vi, at antallet af elementer i listen $L[h(x)]$ må være alle dem som hasher til værdien $h(x)$, altså alle $y$ så $h(y) = h(x)$.\\
I \cref{eq:lin-of-exp} bruger vi Linearity of Expectation.\\
I \cref{eq:1-over-m} benytter vi vores antagelse om at vores hashing function er universel, hvorved uligheden gælder.\\
I \cref{eq:n-over-m} benytter vi, at der i alt er $n$ elementer i $S$.






\subsection{Opbygning af dictionary der altid sikrer konstant opslagstid}
Antag vi har en universel hash funktion $h : U \rightarrow [m]$. hvor $m \geq 1$ er et heltal vi vælger senere. For ethvert $i \in [m]$, definer $S_i = \{ x \in S | h(x) = i \}$ og $n_i = |S_i|$. For ethvert sæt $S_i$ er køretiden $O(1)$ hvis det er tomt eller kun indeholder et element. Men hvis mange elementer hasher til samme index vil det tage lang tid.\\

Lad os definere $C_h$ til at være det totale antal nøgler i $S$ der kolliderer under hashfunktionen $h$. Da vil $C_h = \sum_{i \in [m]} \binom{n_i}{2}$ (f.eks. vil det for ét indeks $i$ give $0$ når $n_i = 0$ eller $n_i = 1$, $C_h = 1$ når $n_i = 2$, $C_h = 3$ når $n_i = 3$ osv.)\\


Vi ønsker en hashfunktion så $C_h = 0$, hvorved der for alle sæt $S_i$ gælder de er tomme eller kun indeholder ét element.\\

Vi har pr. definition af $C_h$ følgende, hvor vi stadig lader indikatorvariablen $I(y)$ betegne $[h(x) = h(y)]$:
\begin{align}
  C_h = \sum_{\mathclap{\substack{ \{x, y\} \in S\\ x \neq y}}} \ I(y)
\end{align}

Da vi antager $h$ er universel har vi, at $\E{I(y)} \leq 1/m$ når $x \neq y$:
\begin{align}
  \E{C_h}
  &= \E{ \ \ \sum_{\mathclap{\substack{ \{x, y\} \in S\\ x \neq y}}} \ I(y)} \nonumber \\
  &= \sum_{\mathclap{\substack{ \{x, y\} \in S\\ x \neq y}}} \ \E{I(y)} \label{eq:lin-of-expp} \\
  &\leq \binom{n}{2} * \frac{1}{m} \label{eq:n-choose-2} \\
  &= \frac{n!}{2!(n-2)!} * \frac{1}{m} \nonumber \\
  &= \frac{1}{2m} * \frac{n!}{(n-2)!} \nonumber \\
  &= \frac{n(n-1)}{2m} \label{eq:binom-to-frac}
\end{align}

I \cref{eq:lin-of-expp} bruger vi linearity of expectation.\\
I \cref{eq:n-choose-2} har vi $\binom{n}{2}$ da $n = |S|$ og vi vælger to elementer. Forventningen $1/m$ er givet jf. vores antagelse.\\
Frem til \cref{eq:binom-to-frac} er det simpel købmandsregning.\\


Nu kan vi bestemme:
\begin{align}
  \P{C_h > \frac{n(n-1)}{m}}
  &= \P{C_h > 2 \E{C_h}} \nonumber \\
  &\leq \P{C_h \geq 2 \E{C_h}} \nonumber \\
  &\leq \frac{\E{C_h}}{2 \E{C_h}} \label{eq:benyt-markov} \\
  &= \frac{1}{2} \nonumber
\end{align}

Hvor vi i \cref{eq:benyt-markov} benytter Markovs ulighed.\\

Hvis vi vælger en ny hashfunktion $h$ hver gang $C_h > n(n-1)/m$ vil det forventede antal gange vi skal gøre det være højest $1/(1/2) = 2$, da det er geometrisk distribueret.\\


Hvis vi i stedet vælger $m = n^2$ vil det forventede antal hashfunktioner $h$ vi skal prøve før vi får $C_h = 0$ stadig være højest 2. Det ser vi ved at indsætte $m = n^2$, hvorved vi får:
$$
C_h \leq \frac{n(n-1)}{n^2} < 1
\quad\quad \rightarrow \quad\quad
C_h = 0
$$
efter forventet højest 2 forsøg. Det gælder da vi kun arbejder med heltal.\\

Med denne metode får vi altså en dictionary med 0 kollisioner, og derved $O(1)$ opslagstid. Hver eneste forsøg på at finde en passende $h$ tager $O(n+m)$ tid, så det vil også være den forventede køretid.\\

Til gengæld bruger vi $\Theta(n^2)$ plads.






\subsection{Bestemmelse af pladsforbrug for 2-level hashing}
Lad os nu vælge $m = n$. Nu kan vi finde en hash funktion $h$ så $C_h \leq n$ i forventet højest 2 forsøg (da $C_h \leq \frac{n(n-1)}{n} < n$ jf. hvad vi viste lige før).\\

Det andet niveau består af en 1-level hash tabel for hvert $S_i$ hvor $|S_i| = n_i \geq 1$. Så for hvert $i \in [n]$ hvor $n_i \geq 1$ lad $h_i : U \rightarrow [m_i]$ være en universel hashfunktion hvor vi definerer $m_i = n_i^2$. Hvis der sker nogle kollisioner prøver vi igen med en ny $h_i$. På samme måde kan vi beregne at vi forventet højest skal bruge 2 forsøg før vi finder en $h_i$ der opfylder dette.\\


Da er pladsforbruget:
\begin{align}
  O \p{ 1 + n +m + \sum_{i \in [m]} \p{ 1 + n_i + m_i }   }
  &= O \p{ n + \sum_{i \in [m]} \p{ 1 + n_i + n_i^2 }  } \label{eq:smid-ligegyldige-ting-vaak} \\
  &= O \p{ n + \sum_{i \in [m]} \p{ 2 n_i + n_i ( n_i - 1) } } \nonumber \\
  &= O \p{ n + \sum_{i \in [m]} \p{ 2 n_i + 2 \binom{n_i}{2}}  } \label{eq:benyt-binom-def} \\
  &= O \p{ n + 2 \sum_{i \in [m]} \p{ n_i + \binom{n_i}{2} }} \nonumber \\
  &= O(n + 2n + 2C_h) \nonumber \\
  &= O(n) \nonumber
\end{align}

I \cref{eq:smid-ligegyldige-ting-vaak} benytter vi $n = m \geq 1$ og vores værdi for $m_i$.\\
I \cref{eq:benyt-binom-def} benytter vi igen $\binom{n}{2} = n(n-1)/2$.\\



Hver eneste forsøg på at finde en $h_i$ tager $O(n_i + m_i)$ tid, så den forventede tid det tager at finde en passende $h_i$ er også $O(n_i + m_i)$. Det har vi lige bestemt er $O(n)$, så vi får altså en samlet køretid for denne procedure (med 1-level delen også) på $O(n)$.

\end{document}
