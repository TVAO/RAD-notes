\section{Set Balancing}
\subsection{Problem}

Givet en $n \times n$ $0/1$-matrix $\mathbf A$, find vektor $\mathbf b \in \curly{-1, 1}^n$ så $||\mathbf{Ab}||_\infty$ (udtales infinity-normen) bliver så lille som mulig. Eksempelvis kunne vi have:
\begin{alignat*}{5}
  &\begin{bmatrix}
  0 & 1 & 1\\
  1 & 0 & 1\\
  1 & 1 & 1
  \end{bmatrix}
  &&\begin{bmatrix}
  -1\\
  1\\
  1
  \end{bmatrix}
  =
  &&\begin{bmatrix}
  2\\
  0\\
  1
  \end{bmatrix}
  \quad\quad &&\rightarrow \quad\quad
  &&||\mathbf{Ab}||_\infty = 2\\
  %%%%%%%%%%%%%%%%%%%%%%%
  &\quad\quad \mathbf A
  &&\quad \ \mathbf b
  && \ \mathbf{Ab}\\
  %%%%%%%%%%%%%%%%%%%%%%%
  &
  &&\begin{bmatrix}
  -1\\
  -1\\
  1
  \end{bmatrix}
  =
  &&\begin{bmatrix}
  0\\
  0\\
  -1
  \end{bmatrix}
  \quad\quad &&\rightarrow \quad\quad
  &&||\mathbf{Ab'}||_\infty = 1\\
  %%%%%%%%%%%%%%%%%%%%%%%
  &
  &&\quad \ \mathbf b'
  && \ \mathbf{Ab'}\\
\end{alignat*}

Her er f.eks. $||\mathbf{Ab}||_\infty = 2$ da den er defineret som den numerisk største værdi i vektor $\mathbf{Ab}$.\\

Man kan f.eks. forstå problemet som at matrix $\mathbf A$ i søjlerne har en række individer, og i rækkerne har en værdi for om besidder en bestemt egenskab eller ej. Så vil vi gerne dele individerne i to grupper, som skal sikre så jævn fordeling af egenskaberne som muligt. $||\mathbf{Ab}||_\infty$ er da et udtryk for den egenskab der er mest ujævnt fordelt.






\subsection{Algoritme}
Vælg indgang $i$ i $\mathbf b$ til $-1$ med sandsynlighed $1/2$ og ellers $1$ for alle $i$ uafhængigt.






\subsection{Analyse af sandsynlighed for $||\mathbf{Ab}||_\infty \leq 2 \sqrt{2 n \ln n}$}
Betragt række $j$ i $\mathbf A$, $\mathbf A_j$. Antag at $\mathbf A_j$ er på følgende form, som er en gyldig måde at anskue det på så længe vi kun kigger på én række, da vi i vores analyse bare kunne flytte alle 1-taller i $\mathbf A_j$ til venstre og så have de resterende 0-taller til højre:
$$
\mathbf A_j = [ \; \underbrace{1 \; \dots \; 1}_{k} \; \underbrace{0 \; \dots \; 0}_{n-k} \; ]
$$

Alle 0-taller kan vi da ignorere, da de ikke vil bidrage til summen i vores analyse.\\
For $i = 1, \dots, k$, lad da indikatorvariablen $X_i$ være
$$
X_i
=
\begin{cases}
	1 & \text{hvis $b_i = -1$}\\
	0 & \text{ellers (dvs. $b_i = 1$)}\\
\end{cases}
$$

Lad nu
$$
X = \sum_{i=1}^k X_i \quad\quad\quad\quad \mu_X = k * \P{X_i = 1} = \frac{k}{2}
$$


Vi får $\mathbf A_j \mathbf b = 0$ når de første $k$ indgange i $\mathbf b$ er præcis ligeligt fordelte mellem at være $-1$ og $1$:

\[
  \setlength{\arraycolsep}{0pt}
  \mathbf{b} =
    \left[
      \begin{array}{c}
        \begin{array}{c}
          -1 \\ -1 \\ 1 \\ 1
        \end{array} \\
        \begin{array}{c}
          \vdots
        \end{array}
      \end{array}
    \right]
  ~ % Some space
  \begin{array}{c}
    \noleftdelimiter
    \vphantom{\begin{array}{c}
      -1 \\ -1 \\ 1 \\ 1
    \end{array}}
    \right\} k \\
    %\noleftdelimiter
    \vphantom{\begin{array}{c}
      \vdots
    \end{array}}
    %\right\} n - k
  \end{array}
\]


Altså får vi:

\begin{align}
  \P{\mathbf A_j \mathbf b = 0} &= \P{X = \frac{k}{2}} \nonumber \\
  \P{\mathbf A_j \mathbf b = 2} &= \P{X = \frac{k}{2} - 1} \nonumber \\
  &\vdots \nonumber \\
  \P{\mathbf A_j \mathbf b = 2c} &= \P{X = \frac{k}{2} - c} \nonumber \\
  &\Updownarrow \nonumber \\
  \P{\mathbf A_j \mathbf b > 2c} &= \P{X < \frac{k}{2} - c} \label{eq:moenster-prob}
\end{align}
Da hvis vi gør $c$-værdien meget stor, så betyder det at vi har meget få $-1$'ere, dvs. $\mathbf A_j \mathbf b$ bliver stor.\\

Nu benytter vi \cref{eq:chernoff-2}, hvor der skal gælde $0 < \delta \leq 1$:
\begin{align}
  \P{X < (1 - \delta) \mu } < e^{- \frac{\delta^2 \mu}{2} } = e^{- \frac{\delta^2 k}{4} } = \frac{1}{n^2} \label{eq:set-balance}
\end{align}

Når vi vælger $\delta^2 = \frac{8 \ln n}{k} \Longrightarrow \delta = 2 \sqrt{\frac{2 \ln n}{k}}$.\\

Skriver vi videre på venstresiden af \cref{eq:set-balance} får vi:
\begin{align}
  \P{X < (1 - \delta) \mu }
  &= \P{X < \frac{k}{2} - \delta \mu } \label{eq:indsaat-vaardi-for-mu} \\
  &= \P{\mathbf A_j \mathbf b > 2 \delta \mu} \label{eq:brug-tidligere-eq} \\
  &= \P{\mathbf A_j \mathbf b > k \delta} \nonumber \\
  &= \P{\mathbf A_j \mathbf b > 2 \sqrt{2 k \ln n}} \label{eq:brug-vaardi-for-delta} \\
  &\geq \P{\mathbf A_j \mathbf b > 2 \sqrt{2 n \ln n}} \label{eq:saat-k-til-n}
\end{align}

I \cref{eq:indsaat-vaardi-for-mu} ganger vi parentesen ud og indsætter vores værdi for $\mu$ på leddet uden $\delta$.\\
I \cref{eq:brug-tidligere-eq} benytter vi hvad vi fandt frem til i \cref{eq:moenster-prob}.\\
I \cref{eq:brug-vaardi-for-delta} indsætter vi vores værdier for $k$ og $\delta$.\\
I \cref{eq:saat-k-til-n} benytter vi, at der ikke kan være flere 1-taller end længden af vektor $\mathbf b$, altså $k \leq n$. Herved bliver uligheden sværere at opfylde, så sandsynligheden falder.\\


Hermed har vi altså vist, at for en række $\mathbf A_j$ har vi følgende, hvor medfører pilen gælder ved symmetri:
$$
\P{\mathbf A_j \mathbf b > 2 \sqrt{2 n \ln n}} < \frac{1}{n^2}
\Longleftrightarrow
\P{\mathbf A_j \mathbf b < - 2 \sqrt{2 n \ln n}} < \frac{1}{n^2}
$$

Vi kan nu bruge union bound til at få:
$$
\P{|\mathbf A_j \mathbf b| > 2 \sqrt{2 n \ln n}} < \frac{2}{n^2}
$$

Går vi nu videre fra bare at kigge på en række $\mathbf A_i$ til hele matrix $\mathbf A$ får vi, igen ved union bound, at:
\begin{align*}
  \P{||\mathbf{Ab}||_\infty > 2 \sqrt{2 n \ln n}}
  = \P{ \bigcup_{j=1}^n  \curly{|\mathbf A_j \mathbf b| > 2 \sqrt{2 n \ln n}} }
  \leq n * \frac{2}{n^2}
  = \frac{2}{n}
\end{align*}



Altså har vi nu vist, at med sandsynlighed $\geq 1 - \frac{2}{n}$ gælder $||\mathbf{Ab}||_\infty \leq 2 \sqrt{2 n \ln n}$.
