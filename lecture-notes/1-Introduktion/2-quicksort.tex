\section{Randomized QuickSort (RQS)}

Vi starter med at definere ''rank'' $i$ af et element, som vi lader være dets position $S_{(i)}$ i det endelige sorterede array (1-indeksering), altså:\\

$S_{(i)}$: Element med rank $i$ (givet 1-indeksering)\\

F.eks.:
\begin{verbatim}
input:  [3, 0, 5, 2] (usorteret array)
output: [0, 2, 3, 5] (sorteret array)
\end{verbatim}
Så vil eksempelvis $S_{(2)} = 2$ og $S_{(4)} = 5$\\


For alle $1 \leq i < j \leq n$ har vi, at $X_{ij} = 1$ hvis $S_{(i)}$ og $S_{(j)}$ sammenlignes af RQS, og 0 ellers. Vi kan gøre det med en simpel indikatorvariabel, da to elementer aldrig kan blive sammenlignet mere end én gang.\\

Da får vi følgende, hvor startgrænsen $j > i$ på det andet sumtegn gælder da sammenligningerne er symmetriske.
$$
X = \sum_{i=1}^n \sum_{j>i} X_{ij}
$$
\textit{Mål: Find øvre grænse for $\e[X]$}

Nu bruges \cref{eq:linearity}, linearity of expectation:
\begin{align} \label{eq:exp-X}
\e[X] = \e \square{ \sum_{i=1}^n \sum_{j>i} X_{ij}} = \sum_{i=1}^n \sum_{j>i} \e[X_{ij}] = \sum_{i=1}^n \sum_{j>i} \P[X_{ij} = 1]
\end{align}

Herefter analyserer vi algoritmen:\\
Den fungerer på følgende måde, at hvis vi f.eks. vælger $S_{(4)}$ som pivot i et array med seks elementer vil vi have:
\begin{align} \label{eq:eksempel}
\square{S_{(1)}, S_{(2)}, S_{(3)},} \underbrace{S_{(4)}}_{\text{pivot}}, \square{ S_{(5)}, S_{(6)}}
\end{align}
Herefter vil $S_{(4)}$ aldrig blive sammenlignet med nogle elementer igen, og algoritmen ville nu kigge på de to delarrays der er lavet hver for sig.


Lad os skrive det op mere generelt, hvor vi har to indekser $i,j$ hvor $1 \leq i < j \leq n$. Vi kan da skrive vores array op således:
$$
S_{(1)}, S_{(2)}, S_{(3)}, \cdots, S_{(i)}, \cdots, S_{(j)}, \cdots, S_{(n-1)}, S_{(n)}
$$

Nu vælges en pivot $S_{(k)}$. Da har vi tre mulige cases som beskrevet nedenfor for hvor $k$ vælges:

\begin{enumerate}
	\item $k < i$ eller $k > j$:\\
	$S_{(i)}$ og $S_{(j)}$ bliver ikke sammenlignet i denne iteration af rekursionstræet
	\item $i < k < j$:\\
	$S_{(i)}$ og $S_{(j)}$ sammenlignes \textit{aldrig}. Det skyldes at de vil blive delt ind i hver deres subarray, se evt. \cref{eq:eksempel}.
	\item $k=i$ eller $k=j$:\\
	$S_{(i)}$ og $S_{(j)}$ sammenlignes netop én gang (og herefter aldrig igen).
\end{enumerate}

Vi kan nu i vores analyse ignorere case 1, da $S_{(i)}$ og $S_{(j)}$ ikke sammenlignes her, men vi stadig går et niveau dybere i vores rekursionstræ og der derfor er mulighed for at de sammenlignes senere.\\

Vi ser generelt for case 2 og 3, at de gælder når $i \leq k \leq j$. Case 3 giver os 2 mulige udfald for at de bliver sammenlignet. Antallet af mulige udfald i alt er længden fra $i$ til $j$, som må være $j - i + 1$. Altså får vi:
\begin{align} \label{eq:prob-xij}
\P[X_{ij} = 1] = \frac{2}{j - i + 1}
\end{align}


Vi benytter nu dette til at regne videre på \cref{eq:exp-X}:
\begin{align}
\e[X] &= \sum_{i=1}^n \sum_{j>i} \P[X_{ij} = 1] \nonumber \\
      &= \sum_{i=1}^n \sum_{j>i} \frac{2}{j - i + 1} \nonumber \\
      &= 2 \sum_{i=1}^n \sum_{j>i} \frac{1}{j - i + 1} \nonumber \\
   &\leq 2 \sum_{i=1}^n \sum_{k=1}^n \frac{1}{k} \label{eq:mindre} \\
      &= 2n \sum_{k=1}^n \frac{1}{k} \nonumber \\
      &= 2n H_n \label{eq:harmonic} \\
      &= O(n \lg n) \nonumber
\end{align}
Ovenfor gælder uligheden ved \cref{eq:mindre}, da $j-i+1 \geq 2$ (da $j$ altid minimum er 1 større end $i$), og $j$ ikke vil gå ''længere ud'' end til $n$.\\
I \cref{eq:harmonic} benytter vi hvad der kaldes det $n$'te Harmoniske tal, som er defineret som $H_n = \sum_{i=1}^n (1/i) = \ln n + \Theta(1)$.\\

Herved kommer vi altså frem til, at $\e[X] = O(n \lg n)$. Da $\e[X]$ er det forventede antal sammenligninger som Randomized QuickSort udfører, må køretiden ligeledes være dette for algoritmen.
